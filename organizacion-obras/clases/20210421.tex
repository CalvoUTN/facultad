\documentclass[../main.tex]{subfiles}

\begin{document}

\subsection{Materiales}

\href{https://youtu.be/9XTnfhrvx5Y}{Link a clase 20210421}

Lo primero es necesario encontrar que los costos de los materiales tengan tres
caracteristicas:

\begin{itemize}
  \item Sin I.V.A.
  \item Puesto en obra.
  \item Unidades de cotización.
\end{itemize}

Es importante también que en la cantidad de material que se coloca sean considerados
los \textbf{desperdicios}.

\subsection{Equipos}

\subsubsection{Máquinas y herramientas que no constituyen equipo}

Se consideran ciertos elementos que no se consideran, tales como:

\begin{itemize}
  \item Bombas.
  \item Trituradora de ladrillos.
  \item Moledora de obra.
  \item Hormigoneras tales como la \textit{perita}.
  \item Cizalla para cortar hierro.
  \item Compresor de aire.
  \item Sierras, como circular.
  \item Amoladora y taladro.
  \item Inglete.
  \item Aparejos.
\end{itemize}

Además otras herramientas varias como maceta, martillos, piquetas, cucharas de
albañil, hachuelas, tenazas de armador, barreta, corta fierros y corta puntas.

\subsubsection{Que sí son equipos}

Es toda maquinaria y/o conjunto de maquinaria que trabaje directamente en la
obra para la realización de un ítem de la misma.

Podemos clasificarlos segun varios tipos, como segun como se propulsan, tipo de
obra. 

\begin{description}
  \item[Movimiento de suelos] excavadoras, topadoras, cargadoras, retro, motoniveladora.
  \item[Transporte de suelos] camiones, volquetas, grúas, plumas, traíllas.
  \item[Hormigon] silos, plantas de elaboración, dosificadoras, etc.
  \item[Asfalto] plantas elaboradoras, dosificadoras, terminadoras, camiones.
  \item[Compactadores] pata de cabra, etc.
  \item[Mezcla de materiales] hormigoneras, etc.
  \item[Auxiliares] compresores, perforadores, compresivos, etc.
  \item[Puentes] grúas, equipos de postesados, equipos de izamiento, etc.
\end{description}

\subsection{Costo de equipo}

Para poder determinar el costo debemos debemos determinar el costo individual de
cada máquina. Entonces debemos determinar tanto el \textit{costo horario $\$/h$},
y el \textit{rendimiento $h/ h^3$}. 

Este costo incluye tres cosas:

\begin{description}
  \item[Equipo] se debe considerar la \textbf{amortización e intereses}. Esto es 
    el monto a reservar periódicamente para que al final de la vida útil de la 
    máquina se pueda sustituir por una nueva. 

    El concepto de \textit{vida útil} es el período de actividad de un equipo a 
    partir del cual los gastos de conservación y reparación superan la 
    amortización, aunque también es necesario afectarla a los avances tecnologicos.
    El cálculo de amortización se hace como:
    
    \begin{align*}
      A = \frac{\text{Valor nuevo - Valor residual}}{\text{Vida útil}}
    .\end{align*}
\end{description}


\end{document}
