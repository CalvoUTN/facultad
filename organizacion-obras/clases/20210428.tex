\documentclass[../main.tex]{subfiles}

\begin{document}

 \subsection{Repaso}

 La determinación de precios unitarios consiste en determinar el precio de las 
 cantidades unitarias de cada ítem, y requiere un análisis de las cantidades
 de materiales, mano de obra y equipos a emplear en la ejecución de la unidad del
 cómputo y determinar su costo.

 Tiene como como caracteristica lo siguiente:

 \begin{description}
   \item[Aproximado] la determinación del costo directo es aproximado, lo que 
     quiere decir que el análisis de dos personas pueden no ser iguales, debido
     a que las estimaciones de los rendimientos pueden ser muy distintos.

   \item[Específico] es especifico para cata item.

    \item[Dinámico] es algo que se va moviendo a lo largo del tiempo al tener en
      cuenta la inflación.
 \end{description}

 Es importante el desglose de tres aspectos importantes en el análisis: los 
 materiales, los equipos y la mano de obra. El precio saldra desde un método
 inductivo o uno deductivo, utilizando \textit{costos anteriores}, ya que 
 sirven para la determinación del \textbf{rendimiento}.

 \subsection{Ejemplo práctico}
 
\href{https://youtu.be/Wxb7mm8b-uI}{Clase del 20210428}

 Mampostería de ladrillos comunes de $\SI{0.30}{m}$ de espesor con mortero 
 1/8:1:4 (cemento, cal y arena) con el 15\% de agua. Se emplean ladrillos de
 5x12x25. 

\subsubsection{Ladrillos}

 Se requiere determinar la cantidad de ladrillos por $m^3$ de mampostería de 30.
 Para eso con un esquema se diagraman las juntas, y se puede llegar a lo siguiente:

  \begin{align*}
    \text{Cantidad} &= \frac{100}{14} * \frac{100}{7} * \frac{100}{30} \\[5pt]
    \text{Cantidad} &= 340
 .\end{align*}

Tenemos en cuenta un desperdicio de ladrillos de 10\%, por lo que en realidad
consideramos 374 ladrillos.

\subsubsection{Mezcla}

Luego, para considerar la mezcla, debemos restar a $\SI{1}{m^3}$ el volumen de
340 ladrillos, entonces tenemos:

\begin{align*}
  \text{Mezcla} &= \SI{1}{m^3} - 340*0.5*0.13*0.3 \\[5pt]
  \text{Mezcla} &= \SI{0.388}{m^3} = \SI{388}{lt}
.\end{align*}

De igual forma, a este valor hay que afectarlo por un desperdicio, que debemos
considerar en un 10\% por lo que debemos considerar en realidad $\SI{0.4268}{m^3}$.

A esta cantidad debemos contarlo en sus componentes, entonces utilizamos el método
de \textit{coeficientes de aporte}, que consiste en determinar la relación que
existe del volumen aparente de un material y su volumen real. En función de esta
relación podemos considerar el volumen real de los elementos.

Obtenemos los valores del cemento, la cal y la arena. Estos son:

% Table generated by Excel2LaTeX from sheet 'Hoja1'
\begin{table}[htbp]
  \centering
    \begin{tabular}{lr}
    \hline
    \multicolumn{1}{|l|}{Coef. Aporte} & \multicolumn{1}{l|}{Valor} \bigstrut\\
    \hline
    Cemento & 0.45 \bigstrut[t]\\
    Cal hidraulica & 0.5 \\
    Arena mediana & 0.57 \\
    \end{tabular}%
\end{table}%

% Table generated by Excel2LaTeX from sheet 'Hoja1'
\begin{table}[htbp]
  \centering
    \begin{tabular}{rrr|r|}
    \hline
    \multicolumn{1}{|l|}{Material} & \multicolumn{1}{l|}{Volumen ap ($m^3$)} & \multicolumn{1}{l|}{Coef. Aporte} & \multicolumn{1}{l|}{Volumen real ($m^3$)} \bigstrut\\
    \hline
    \multicolumn{1}{|l|}{Cemento} & \multicolumn{1}{r|}{0.125} & 0.45  & 0.05625 \bigstrut\\
    \hline
    \multicolumn{1}{|l|}{Cal hidraulica} & \multicolumn{1}{r|}{1} & 0.5   & 0.5 \bigstrut\\
    \hline
    \multicolumn{1}{|l|}{Arena mediana} & \multicolumn{1}{r|}{4} & 0.57  & 2.28 \bigstrut\\
    \hline
    \multicolumn{1}{|l|}{Agua} & \multicolumn{1}{r|}{0.769} & 1     & 0.769 \bigstrut\\
    \hline
          &       &       & 3.60525 \bigstrut\\
\cline{4-4}    \end{tabular}%
\end{table}%

Luego, haciendo una regla de tres simple, obtenemos la dosificación para encontrar
los volumenes para un metro cúbico como se muestra en \Cref{tab:tablaprimaria}. 
Luego, multiplicando por el peso especifico podemos obtener el valor en 
kilogramos, o pasarlo a litros de ser necesario. Luego podemos obtener el valor
para nuestro caso partícular como se ve en \Cref{tab:tablaparticular}

% Table generated by Excel2LaTeX from sheet 'Hoja1'
\begin{table}[ht]
  \centering
  \caption{Base de cálculo}
    \begin{tabular}{|l|rl|}
    \hline
    \multicolumn{3}{|c|}{Dosificación por metro cúbico} \bigstrut\\
    \hline
    Cemento & 0.035 & $m^3$ \bigstrut\\
\cline{1-1}    Cal hidraulica & 0.277 & $m^3$ \bigstrut\\
\cline{1-1}    Arena mediana & 1.109 & $m^3$ \bigstrut\\
\cline{1-1}    Agua  & 0.213 & $m^3$ \bigstrut\\
    \hline
    \end{tabular}%
  \label{tab:tablaprimaria}%
\end{table}%


% Table generated by Excel2LaTeX from sheet 'Hoja1'
\begin{table}[ht]
  \centering
  \caption{Conversión a volumen particular}
    \begin{tabular}{|r|rl|rl|}
    \hline
    \multicolumn{1}{|l|}{Conversion} & \multicolumn{2}{c|}{Dosificación por metro cúbico} & \multicolumn{2}{c|}{Para 0.4268} \bigstrut\\
    \hline
    1400  & 48.5  & kg    & 20.72 & kg \bigstrut\\
\cline{1-1}    450   & 124.8 & kg    & 53.27 & kg \bigstrut\\
\cline{1-1}    1     & 1.1   & $m^3$ & 0.47  & $m^3$ \bigstrut\\
\cline{1-1}    1000  & 213.3 & litros & 91.04 & litros \bigstrut\\
    \hline
    \end{tabular}%
  \label{tab:tablaparticular}%
\end{table}%


\subsection{Tabla de mano de obra}

Tomando rendimientos de la mano de obra del libro, podemos armar las siguientes
tablas que serían presentadas, como se ve en \Cref{tab:tablafinal}.

% Table generated by Excel2LaTeX from sheet 'Ladrillo'
\begin{table}[ht]
  \centering
  \caption{Tabla oficial}
    \begin{tabular}{|c|c|c|c|r|}
    \hline
    \multicolumn{1}{|l}{Material} & \multicolumn{2}{c}{Cantidad} & \multicolumn{1}{l}{Precio unitario} & \multicolumn{1}{l|}{Precio total} \bigstrut\\
    \hline
    \multicolumn{1}{|l|}{Cemento} & \multicolumn{1}{r}{20.72} & \multicolumn{1}{l|}{kg} &       &  \bigstrut\\
\cline{1-1}\cline{4-5}    \multicolumn{1}{|l|}{Cal hidraulica} & \multicolumn{1}{r}{53.27} & \multicolumn{1}{l|}{kg} &       &  \bigstrut\\
\cline{1-1}\cline{4-5}    \multicolumn{1}{|l|}{Arena mediana} & \multicolumn{1}{r}{0.47} & \multicolumn{1}{l|}{$m^3$} &       &  \bigstrut\\
\cline{1-1}\cline{4-5}    \multicolumn{1}{|l|}{Agua} & \multicolumn{1}{r}{91.04} & \multicolumn{1}{l|}{litros} &       &  \bigstrut\\
    \hline
    \multicolumn{4}{|c|}{Total materiales} &  \bigstrut\\
    \hline
    \multicolumn{1}{|l|}{Mano de obra} & \multicolumn{2}{c|}{Cantidad} & \multicolumn{1}{l|}{Precio unitario} & \multicolumn{1}{l|}{Precio total} \bigstrut\\
    \hline
    \multicolumn{1}{|l|}{Oficial} & \multicolumn{1}{r}{7} & \multicolumn{1}{l|}{h} &       &  \bigstrut\\
\cline{1-1}\cline{4-5}    \multicolumn{1}{|l|}{Ayudante} & \multicolumn{1}{r}{8} & \multicolumn{1}{l|}{h} &       &  \bigstrut\\
    \hline
    \multicolumn{4}{|c|}{Total mano de obra} &  \bigstrut\\
    \hline
    \end{tabular}%
  \label{tab:tablafinal}%
\end{table}%


Si se tubiera un material compuesto, como puede ser el revoque exterior a la cal
completo (ver ejemplo de clase o resuelto en Excel), se debe sumar la cantidad
de cemento, por ejemplo, para cada capa del revoque, para determinar el valor
final de metros cúbicos necesarios por metro cuadrado de revoque.


\end{document}
