\documentclass[../main.tex]{subfiles}

\begin{document}

\section{Certificados de obra}

Un certificado de obra es un documento que se da cada mes, que funciona como una
declaración administrativa del comitente referida al contrato de una obra,
destinada a reconocer un crédito a favor del contratista. 

Otra forma es considerarlos como \textbf{pagos parciales} a cuenta de una
ulterior liquidación de obra con la finalidad de reposición de gastos sin
necesidad de esperar a la finalización total de la misma.

\subsection{Pasos}

Lo primario que se hace es solicitar la inspección, y luego la elaboración de
un acta de medición. La ausencia del representante técnico no impide que el
inspector no realice el acta. Luego de esta se \textbf{firma}.

Si el representante se encuentra disconforme con el acta, esta se debe firmar 
de igual forma y luego se cuentan con \textbf{5 días} para realizar un reclamo
sobre el acta labrada, luego de eso, el contratista tiene \textbf{10 días} para
contestar la misma.

Por último, luego que se dio lugar a lo anterior, se debe \textbf{elaborar el} 
\textbf{certificado}, que será ponerle precio al acta elaborada, utilizando los
costos obtenidos a partir del análisis de precio.

\subsection{Tipos de certificados s/etapa}

Pueden ser \textbf{provisionales o parciales}, que son \textit{acumulados}, es
decir que los certificados nunca se contabilizan solo de lo hecho en el mes,
sino todo lo hecho. Además tenemos \textbf{certificados finales}.

Por otro lado tenemos los \textit{certificados por ampliación} o por otro lado
se pueden tener \textit{certificados de adicionales} cuando lo que se agrega
\textbf{no es más de lo mismo}.

Tenemos también tipos de \textbf{certificados de acopio} o \textbf{desacopio}.
El desacopio será acompañado de un certificado de desacopio. Contendrá las
cantidades de materiales que habiendo sido acopiado hayan sido utilizados en
los trabajos incluidos en el certificado mensual de obra que se trate.

El \textbf{certificado de variaciones de precio} se utiliza para redeterminar
el precio de lo cotizado originalmente. En Entre Ríos se utiliza un software
provincial para poder hacerlo, pero también se puede hacer por índice, por
fórmulas polinómicas o en algunos casos se puede hacer mediante la
redeterminación de precios.






\end{document}
