\documentclass[../main.tex]{subfiles}

\begin{document}

\section{Análisis de precios}

\subsection{Mano de obra }

\href{https://www.youtube.com/watch?v=K7TIwno_aDo}{Link clase 20210414}.
\href{https://www.youtube.com/watch?v=GWOS0LerSx0&list=PLoCwJEKoNRRR7x45YErYqb5XJkEzawFJu&index=7}{Link clase 20210416}.


Este tipo de estudio es importante ya que es necesario para calcular el valor del
rendimiento. Se utiliza la \textit{Ley de la Construcción} como base, donde se
establece que la \textbf{forma de pago} será de forma de jornal y cada quincena.

Se pueden distinguir los siguientes trabajos:

\begin{description}
  \item[Ayudante] es quien colabora, y es capaz de realizar actividades de asistencia,
    como preparar mezclas, alcanzar herramientas, etc.
  \item[Medio ayudante] es capaz de realizar tareas básicas como revoques, etc.
    siempre que un oficial le replantee las cosas.
  \item[Oficial] capaz de realizar cualquier trabajo de albañileria.
  \item[Oficial especializ] capaz de organizar una cuadrilla.
\end{description}

% Table generated by Excel2LaTeX from sheet 'Hoja1'
\begin{table}[htbp]
  \centering
  \caption{Jornales básicos}
    \begin{tabular}{|l|r|}
    \hline
    \multicolumn{1}{|c|}{Funcion} & \multicolumn{1}{c|}{\$/hr} \bigstrut\\
    \hline
    Oficial aprendiz & 273.24 \bigstrut\\
    \hline
    Oficial aprendiz & 232.82 \bigstrut\\
    \hline
    Medio oficial & 214.67 \bigstrut\\
    \hline
    Ayudante & 197.07 \bigstrut\\
    \hline
    \end{tabular}%
  \label{tab:addlabel}%
\end{table}%

Si bien, estos jornales pueden cambiar dependiendo de la zona y se le pueden agregar
más por zona desfavorable. Estos valores dependen de las paritarias que se dan en
\textit{abril} de cada año.

Se debe declarar que días se van a adoptar: de lunes a viernes o de lunes a sabado.
En ambos casos se trabajan 44hs por semana, y se debe declarar en el Ministerio
de Trabajo.

También hay consideraciones para la parte de \textbf{lluvia}. Depende de si la 
persona se presenta a trabajar, se le pagan 2.5h. Si la persona ya esta a disposición
para trabajar, es decir se cambió, etc. se le cobrará 4hs, y por cualquier valor
superior se pagará lo trabajado.

\subsubsection{Análisis de días laborables}

Es importante distinguir entre \textbf{días laborables} en el año, y los \textbf{días}
\textbf{pagos}. 

\subsubsection{Días laborables}

Para encontrar los días laborables, empezamos restando los días del año menos
la cantidad de días sabados y domingos, y restando los días feriados que \textbf{no}
caen en sabados y domingo.

\begin{align*}
  D_l = 365 - S_a * 52 - D_o * 52 - F_e * 15
.\end{align*}

Luego, debemos seguir considerando las vacaciones, que significarán 10 días
de ausencia:

\begin{align*}
  D_l = 365 - S_a * 52 - D_o * 52 - F_e * 15 - V_a * 10
.\end{align*}

Luego se consideran, a partir de datos estadísticos, días por enfermedad (15 a pagar
y 17 ausencias), accidentes (5 a pagar y 5 ausencias), y licencias especiales
(1 día pago y 1 ausencia).

\begin{align*}
  D_l = 365 - S_a * 52 - D_o * 52 - F_e * 15 - V_a * 10 - E_n * 17 - A_c * 5 - L_e * 1 - L_l * 15 
.\end{align*}

Terminado la cuenta, tenemos:

\begin{align*}
  D_l = 198
.\end{align*}

\subsubsection{Días pagos}

Para esto, consideramos los días que no se trabaja pero se pagan. Luego, a esto
le debemos sumar los días laborables, y hallamos la cantidad de días pagos:

\begin{align*}
D_p = D_l + V_a * 14 + F_e * 19 + E_n * 15 + A_c * 5 + L_e * 1 + L_l * 3.6 \\[5pt]
D_p = 198 + 57.6 = 255.6
.\end{align*}

\subsubsection{Coeficiente de productividad}

Conociendo los valores de $D_p$ y $D_{l}$, podemos obtener la relación de que tantos
días se trabajan por los que se pagan. Logramos entonces:

\begin{align*}
 C &= \frac{D_p}{D_l} \\[5pt]
 C &= \frac{255.6}{198} \\[5pt]
 C &= 1.291
.\end{align*}

\subsubsection{Aguinaldo y otros pagos adicionales}

El aguinaldo es un sueldo anual complementario que consiste en un mes de sueldo 
de más por cada 12 meses de trabajo. Esto se consigue de la siguiente forma:

\begin{align*}
  \frac{1}{12} * 1.291 = 10,76 \%
.\end{align*}

Es decir, deberemos pagar un 10.76\% del sueldo. 
Además, se deberán pagar ciertos pagos adicionales para tareas partículares, tales
como:

\begin{itemize}
  \item Colada de hormigón por 20\%.
  \item Trabajo en altura por 40\%.
  \item Trabajo en submuración por 3\%.
  \item Trabajo en zanjas en vía pública por 2\%.
  \item Trabajo en revestimientos por 10\%.
\end{itemize}

\subsubsection{Aportes y otros impuestos}

También es necesario considerar el \textbf{fondo de desempleo}. En la Ley de la
Construcción se aclara que por cada mes, durante el primer año, el empleador
deberá ir depositando un 12\% que el empleado recibe si es echado o renuncia.

Este porcentaje pasa a ser un 8\% luego del primer año, y es la razón por la que
los empleados renuncian y vuelven ser contratados.

Además es necesario agregar retenciones al operario, que serán por el \textit{SUSS},
Sindicato, cuota sindical, y otros, que totalizan un 21.6\% que se le restan al
sueldo del empleado. El SUSS consiste en el aporte para la jubilación, asignación
familar, fondo nacional de empleados y otros aportes.

Por otro lado, existen aportes del empleador, que totalizan un 30.97\%. Vemos que
a efectos del costo, solo nos incide el valor de los aportes. Es entonces que el
Estado, por cada 131\$ pagados se queda con 52.5\$. Además, este valor es importante
afectarlo por el coeficiente calculado.

\subsubsection{Seguros}

Las \textbf{A.R.T.} variarán un monto variable de lo que se pagará. En nuestro 
ejemplo se dará un 9\%, pero en principio depende del rubro de donde se desempeña
la empresa y la categoría de riesgo (que depende de la cantidad de accidentes
del año pasado). Además, este también debe ser afectado por el coeficiente 
calculado. 

Además del monto variable se deberá pagar un monto fijo, que el mínimo será de
0.6\$ por operario. 

Otros también es necesario es el \textbf{seguro de vida}, que son tres:
por la Resolución 1567, por el Convenio Colectivo de Trabajo y por UOCRA.
Este valor, tanto como el monto fijo de la A.R.T. debe ser pasado a porcentaje 
haciendo una consideración de la cantidad de horas trabajadas por semana.

Este valor es de alrededor de 17000\$. Considerando que se trabajan 44 horas
por semana durante el año, se trabajan 1742 horas por año. Entonces, con un
análisis de costo, se puede determinar cuanto será el costo total porcentual, que 
será aproximadamente $4.3\%$. 

\subsubsection{Total de incidencias porcentuales}

Si se suman todas las incidencias que se han calculado, se ve que, para un oficial:

\begin{align*}
  \text{Oficial} &= 2.913 + 3.785 + 4.014 + 1.944 0.418 \\[5pt] 
  \text{Oficial} &= 1.3074
.\end{align*}

Esto significa que por cada peso pagado del nominal es necesario pagar 1.31\$ en
otros conceptos denominados \textbf{cargas sociales}. Resumiendo, estas se encuentran
compuestas por: días pagos no laborables, aguinaldo, pagos adicionales, aportes 
y retenciones, A.R.T y seguros de vida.

Con todo esto se elabora una \textbf{planilla de mano de obra}, que considera
el costo por hora nominal, más la asistencia, y multiplicandolo por las cargas 
sociales, así consiguiendo el \textbf{costo horario de la mano de obra}.

\subsubsection{Rendimiento}

Es la cantidad que una persona puede producir en una hora. Se intenta formular
una tabla como la que sigue:

% Table generated by Excel2LaTeX from sheet 'Hoja1'
\begin{table}[ht]
  \centering
  \caption{Tabla de costos}
    \begin{tabular}{rr|l|r|}
    \hline
    \multicolumn{1}{|l|}{\textbf{MANO DE OBRA}} & \multicolumn{1}{l|}{días} & \$/d   & \multicolumn{1}{l|}{Total} \bigstrut\\
    \hline
    \multicolumn{1}{|l|}{Oficial} & 0.15  & \multicolumn{1}{r|}{34.22} & 5.133 \bigstrut\\
    \hline
    \multicolumn{1}{|l|}{Ayudante} & 0.05  & \multicolumn{1}{r|}{28.38} & 1.419 \bigstrut\\
    \hline
          &       & \textbf{SUBTOTAL} & 6.552 \bigstrut\\
\cline{3-4}    \end{tabular}%
  \label{tab:addlabel}%
\end{table}%

Los valores orientativos del rendimiento en días son normalmente datos estadísticos
de la propia empresa o por métodos de \textit{deducción} o \textit{inducción}.
Para estos es importante el desglose que se hizo en el cómputo.






\end{document}
