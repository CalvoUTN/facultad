\documentclass[../main.tex]{subfiles}

\begin{document}

\section{Unidad temática 3: sismoresistencia}

Para este tipo de problemática nos encontramos con tres tipos de soluciones:

\begin{itemize}
  \item Método simplificado.
  \item Método estático.
  \item Método dinámico.
\end{itemize}

En este caso solo vamos a explorar los dos primeros casos. Vamos a tomar como
guía el reglamento CIRSOC 103.

\subsection{Zonificación y clasificación}

\subsubsection{Zonificación sísmica}

Lo primero que se necesita hacer es determinar la zona sísmica. En el país hay
5, desde 0 (peligrosidad muy reducida) a 4 (peligrosidad muy alta). La misma
puede determinarse a partir de la \textbf{Figura 2.1} (pág 33) del CIRSOC 103.

\subsubsection{Clasificación del sitio}

Luego, es necesario conocer la influencia del suelo. Existen tres tipos de
suelos, con 6 tipos de sub-clasificaciones. En general, se determinan mediante
el uso del ensayo SPT o mediante un ensayo que nos permita obtener la 
velocidad media de la onda de corte $V_{sm}$, o utilizando una correlación 
con el SPT o la resistencia al corte no drenada.

Luego, es posible determinar el tipo de suelo mediante la \textbf{Tabla 2.2} 
(pág 34) del CIRSOC 103. 

\subsubsection{Clasificación de construcción según función}

También es necesario determinar el grupo de la construcción. Según el \textbf{Art 2.4}
del CIRSOC 103 (pág 35), quedan determinados los siguientes grupos:

\begin{description}
  \item[Grupo $A_o$ ($\gamma_r = 1.5$ )]  construcciones que cumplen funciones 
    esenciales. Algunos ejemplos son: sectores radioactivos con potencias 
    mayores a 20MW, depósitos de gases, áreas de aeropuertos, hospitales, 
    centros policiales y bomberos, centrales de comunicación, centrales de 
    energía de emergencia y servicios sanitarios básicos.
  \item[Grupo A ($\gamma_r = 1.3$)] normalmente son edificios de servicios médicos,
    estaciones de radio y de televisión, centrales telefónicas, oficinas de
    correos, edificios gubernamentales, escuelas, colegios, universidades,
    cines, teatros, estadios, templos, terminales de transporte, grandes 
    comercios e industrias, museos, bibliotecas, centrales de energía, plantas
    de bombeo, etc.
  \item[Grupo B ($\gamma = 1$] construcciones destinadas a viviendas unifamiliares
    o multifamiliares, hoteles, comercios, entre otros.
  \item[Grupo C ($\gamma_r = 0.8$)] construcciones aisladas con ocupación menor a 10,
    tales como depósitos, establos, silos, etc.
\end{description}

\subsubsection{Excepción para Zona 0}

Para todos los edificios que no sean del grupo $A_o$ que se encuentre en la 
zona, no es aplicable el reglamento si se cumple alguna de las siguientes
condiciones:

 \begin{enumerate}
  \item Las construcciones de hasta 3 pisos o de 12m de altura.
  \item Estructuras de más de 12m que han sido diseñadas al viento, donde
    se cumplen las siguientes tres condiciones de forma simultanea:
    
    \begin{itemize}
      \item Han sido verificadas a las dos direcciones principales.
      \item La resultante en cada dirección es mayor o igual al 1.5\% del peso
        de la estructura.
       \item El punto de aplicación de la fuerza resultante de la acción del
         viento se encuentra aproximadamente coincidente o por encima del 
         centro de gravedad de la construcción.
    \end{itemize}
\end{enumerate}

\subsection{Espectro de diseño}

Lo primero que se debe hacer es la determinación del \textbf{período fundamental}
de la estructura. En todo caso, lo primero que se verifica, según \textbf{Art 6.2.3}
(pág 67) es la siguiente condición:

\begin{align*}
  T \leq C_u * T_a
.\end{align*}

Lo primero que se determina es $C_u$, según  \textbf{Tabla 6.1} (pág 68), que se
reproduce:

\begin{center}
  \begin{tabular}{|c|c|}
    \hline
    $a_s$ & $C_u$ \\
    \hline
    $\geq $ 0.35 & 1.4 \\
    \hline
    0.25 & 1.45 \\
    \hline
    0.15 & 1.60 \\
    \hline
    $\leq $ 0.08  & 0.5 \\
    \hline
 \end{tabular} 
\end{center}

\subsubsection{Período aproximado}


Luego, según \textbf{Art 6.2.3.2} (pag 68) es posible aproximarlo mediante
la siguiente formula:

\begin{align*}
  T_a = C_f * H^x
.\end{align*}

Donde el valor de $C_f$ y $X$ quedan determinados según la \textbf{Tabla 6.2}
(pág 69), pero para lo que se debe primero determinar lo que se describe en
\Cref{ax}.

\subsubsection{Valores de $a_x$,  $C_a$ y  $C_v$}\label{ax}

Los valores mencionados son determinados en la \textbf{Tabla 3.1} (pág 50) del
capitulo 3 del CIRSOC 103. Se tiene como consideración:

\begin{align*}
  N_a &= 1 \\[5pt]
  N_v &= 1.2 \\[5pt]
  T_2 &= \frac{C_v}{2.5*C_a} \\[5pt]
  T_1 &= 0.2 * T_2
.\end{align*}

Donde también queda el valor $T_3$ determinado según la  \textbf{Tabla 3.2},
que se reproduce a continuación:

\begin{center}
  \begin{tabular}{|c|c|}
    \hline
    \textbf{Zona sísmica} & $T_3$ (s) \\
    \hline
    4 & 13 \\
    \hline
    3 & 8 \\
    \hline
    2 & 5 \\
    \hline
    1 & 3 \\
    \hline
  \end{tabular} 
\end{center}

Es importante la determinación de $S_d$, que es la ordenada espectral o 
seudoaceleración según la zona sísmica y tipo de suelo. Otro valor que debemos
obtener es el de $a_s$.

\subsubsection{Espectros de diseño}

Luego que se sepa el valor de período $T$, es necesario determinar $S_a$ según
las expresiones del  \textbf{Art. 3.5.1} (pág 49), que se desarrollan a
continuación:

\begin{align*}
  S_a &= C_a * (1 + 1.5 * T / T_1  \hspace{0.5cm} \text{para } T \leq T_1\\[5pt]
  S_a &= 2.5*C_a  \hspace{0.5cm} \text{para } T_1 \leq T \leq T_2  \\[5pt]
  S_a &= C_v / T  \hspace{0.5cm} \text{para } T_2 \leq T \leq T_3  \\[5pt]
  S_a &= C_v * T_3 /T² \hspace{0.5cm} \text{para } T \leq T_3 
\end{align*}


\subsubsection{Acciones gravitatorias a considerar}

Se deben considerar en las acciones sísmicas horizontales las cargas permanentes
y una fracción de las cargas variables o de servicio. La acción a considerar
es mediante la siguiente expresión, del \textbf{Art. 3.6} (pág 51)

\begin{align*}
  W_i = D_i + \Sigma f_1 *L_i + f_2 * S_i
.\end{align*}

Donde los valores de $f_1$ y $f_2$ quedan determinados mediante la \textbf{Tabla 3.3}
(pág 51) según la carga de ocupación. Para vivienda, normalmente se consideran:

\begin{align*}
  f_1 &= 0.25 \\[5pt]
  f_2 &= 0.2
.\end{align*}




\subsection{Método simplificado}

Para poder utilizar éste método se deben cumplir ciertos requisitos. Lo que 
se debe cumplir es:

\begin{enumerate}
  \item La relación de altura mínima del rectángulo que circunscribe la planta
    es menor o igual que \textbf{2}.
  \item La relación entre el lado mayor y el lado menor del rectángulo que
    circunscribe la planta es menor o igual que \textbf{2}.
  \item En alguna dirección existen al menos \textbf{2} muros exteriores
    resistentes a fuerzas horizontales paralelos o casi paralelos que están
    conectados a las losas o diafragmas un mínimo de \textit{0.5} de longitud
    de la planta en la dirección de esos muros.
  \item En la dirección estudiada existe al menos un muro resistente a fuerzas
    horizontales que está unido a las losas o diafragmas en al menos \textbf{0.8}
    de la longitud de la planta en esa dirección o dos muros conectados un 
    mínimo de \textbf{0.5} de esa longitud.
  \item Los muros mencionados en los puntos anteriores son contiguos en toda
    la altura de la construcción, con una longitud al menos de 1.5 de su altura
  \item La construcción tiene hasta \textbf{2} pisos y hasta \textbf{7m} de altura.
  \item La distancia entre el centro de gravedad de las secciones horizontales
    de los muros resistentes y el centro de gravedad de las masas de cada
    nivel es igual a la mitad de la distancia entre los muros descritos en 3).
\end{enumerate}


\subsubsection{Coeficiente sísmico de diseño}

A partir de la \textbf{Tabla 4.1} (pág 56), podemos determinar el coeficiente
de la zona sísmica. La misma se resume a continuación:

\begin{center}
  \begin{tabular}{|c|c|}
    \hline
    \textbf{Zona sísmica} & $C_n$ \\
    \hline
    1 & 0.23 \\
    \hline
    2 & 0.38 \\
    \hline
    3 & 0.44 \\
    \hline
    4 & 0.5 \\
    \hline
  \end{tabular} 
\end{center}

Con esto podemos determinar el coeficiente de diseño sísmico de diseño para
clases A, B, C y D con la siguiente formula:

\begin{align*}
  C = C_n * \gamma_r
.\end{align*}

\subsubsection{Resultante de fuerzas horizontales}

Se debe obtener una resultante de fuerzas horizontales equivalente a la acción
sísmica operante según la dirección analizada. Esto se logra como se describe 
en la \textbf{Sección 4.2.2} (pág 57):
\begin{align*}
  V_o = C * W
.\end{align*}

Donde $W$ es el peso de la estructura.

Luego, este esfuerzo de corte $V_o$ es comparado con la capacidad a corte de los
muros para cada dirección.o

\subsubsection{Deformaciones}

Según el \textbf{Artículo 4.3} (pág 57), no es necesario estudiar las
deformaciones de las construcciones comprendidas dentro de esta verificación.
Lo único que especifica es que la distancia mínima entre cualquier parte de
la construcción y el plano medio del espacio de separación debe ser:
\begin{align*}
  Y_k \geq \SI{2.5}{cm}
.\end{align*}

\subsection{Método estático}

En éste método es necesario determinar los esfuerzos horizontales. Lo primero
es determinar los esfuerzos de corte en la base, que se logra como:

\begin{align*}
  V_o &=  C * W \\[5pt] 
  W &= \Sigma W_i
.\end{align*}

El \textbf{coeficiente sísmico de diseño} se debe calcular según \textbf{Art 6.2.2}
(pág 67), con las posibilidades:

\begin{align*}
  C = 2.5 * C_a * \gamma_r / R \hspace{0.5cm} &\text{para } T \leq T_2 \\[5pt]
  C = S_a * \gamma_r / R \hspace{0.5cm} &\text{para } T \geq T_2 \\[5pt]
  C \geq 0.8 * a_s * N_v / R \hspace{0.5cm}&\text{para zonas sísmicas 3 y 4}  \\[5pt]
  C \geq  0.11 * C_a * \gamma_r \hspace{0.5cm} &\text{para zonas sísmicas 0, 1 y 2}
.\end{align*}

El mismo se hace luego de determinar el $T$ adoptado, como se explicará en
\Cref{periodo}.


\subsubsection{Distribución de fuerzas horizontales}

Se puede determinar la fuerza sísmica horizontal $F_k$, que se aplica en el
baricentro de la carga gravitatoria $W_k$ ubicada en el nivel $k$ con la
expresión dada en el  \textbf{Art 6.2.4.1} (pág 69).

\begin{align*}
  F_k = \frac{W_k * h_k * V_o}{\Sigma W_i * h_i}
.\end{align*}

\subsubsection{Determinación del periodo}\label{periodo}


% COMPLETAR VIENDO CLASE. CALCULO NO APROXIMADO DE PERIODO. 
% Usa formula C.6.1 de los Comentarios del CIRSOC 103.

\subsubsection{Factor de reducción}

Según el \textbf{Art 5.1} del CIRSOC 103 (pág 59), se puede utilizar un factor
de reducción $R$ que toma el comportamiento en estado último de la construcción.
Lo mismo se hace mediante la  \textbf{Tabla 5.1} (pág 60). En los casos vistos,
normalmente se considera para pórticos un $R=7$.




\end{document}
