\documentclass[../main.tex]{subfiles}

\begin{document}

Las propiedades deseables del asfalto son:

\begin{itemize}
  \item Alta elasticidad a elevadas temperaturas.
  \item Suficiente ductilidad a bajas temperaturas.
  \item Basa susceptibilidad al cambio de temperatura.
  \item Bajo contenido de parafina.
  \item Buena adhesión y cohesión.
  \item Alta resistencia al envejecimiento.
\end{itemize}

Esto nos permite tener un \textbf{asfalto ideal}.

\subsection{Ensayo de consistencia}

Estos ensayos miden la dureza del material y como varía con con respecto
a la temperatura y cargas.

Entre estos encontramos los siguientes.

\begin{description}
  \item[Ensayo de fragilidad Fraass] 
  \item[Ensayo de penetración] se realiza a 25ºC, y nos sirve para medir la
    consistencia del asfalto.
  \item[Ensayo de punto de ablandamiento] nos permite determinar la temperatura
    de ablandamiento.
  \item[Ensayo de viscosidad] 
\end{description}

Otro índice que puede ser interesante es el \textit{indice de penetración de
Pfeiffer y Van Doormaal}, que indica la susceptibilidad térmica del asfalto.
Dependiendo del valor que toma puede ser interpretado el tipo de desempeño
que tendrá, donde un valor entre -1 y 1 es un desempeño moderado.

\subsection{Ensayos de durabilidad}

Se ensaya la capacidad que presenta el asfalto de mantener las propiedades
cohesivas y cementantes con el paso del tiempo, y ciertos agentes tales como
agua, rayos ultravioleta, oxígeno, etc. 

El proceso de mezclado y compactación, a pesar de su corta duración, es crítico.
Es importante controlar las temperaturas y la película de asfalto, que puede
provocar la pérdida de volátiles.

Algunos ensayos son los de \textbf{ensayo de película delgada}.

\subsection{Base asfáltica}

Esta puede ser conformada por asfalto más otro aditivo, tales como:

\begin{itemize}
  \item Con solvente, para conformar asfalto cortado.
  \item Con agua más emulsificante, para formar emulsión asfáltica.
  \item Destilación directa, para conformar cemento asfáltico.
  \item Con modificadores.
\end{itemize}

\subsubsection{Asfalto cortado}

Según la volatizad del solvente y su viscosidad puede tener distintas 
velocidades de curado. En este tipo de asfalto se lo ensaya a consistencia,
seguridad, contenido de asfalto y propiedades del residuo.

\subsubsection{Emulsiones asfálticas}

Son una dispersión estable de pequeños glóbulos de asfalto en agua, que se 
logra estabilizar mediante un emulsificador, que puede ser aniónicas o
catiónicas.

Se da una etapa de \textbf{estabilidad y quiebre}, donde luego de un estado
inicial de dispersión se genera una floculación y coagulación de asfalto, dada
la evaporación del agua.

La velocidad de rotura de la emulsión dependerá de la afinidad del agregado y
emulsión, el tipo de emulsificador y factores climáticos.

\section{Caminos de bajo tránsito}

Son la trama vial donde el tipo de calzada se compone de ripio o de tierra.
Estos caminos conforman una red de servicio a la producción agroganadera y
otros.

Desde el punto de vista de diseño, tiene como ventaja que es de bajo costo
inicial de la estructura, pero se debe buscar una transitabilidad bajo cualquier
condición climática, y conservación simple y de bajo costo.

Se tiene como objetivo, cuando se diseña uno de estos caminos, lo siguiente:

\begin{itemize}
  \item Evitar conflictos del uso de suelo.
  \item Controlar el agua superficial sobre el camino, y mantener los 
    desagües naturales, controlando erosión.
  \item Estabilizar la superficie de rodamiento.
  \item Reducir el desperdicio de materiales.
  \item Impermeabilizar y alargar la vida útil del camino.
\end{itemize}

Luego, se deben hacer consideraciones dependiendo del clima el índice plástico
recomendado, y la necesidad de mejoramientos, sea con estabilización con cal
o cemento.

Los \textbf{materiales naturales aptos} que pueden ser utilizados son el
suelo calcáreo y ripio.



\end{document}
