\documentclass[../main.tex]{subfiles}

\begin{document}

Las propiedades deseables del asfalto son:

\begin{itemize}
  \item Alta elasticidad a elevadas temperaturas.
  \item Suficiente ductilidad a bajas temperaturas.
  \item Basa susceptibilidad al cambio de temperatura.
  \item Bajo contenido de parafina.
  \item Buena adhesión y cohesión.
  \item Alta resistencia al envejecimiento.
\end{itemize}

Esto nos permite tener un \textbf{asfalto ideal}.

\subsection{Ensayo de consistencia}

Estos ensayos miden la dureza del material y como varía con con respecto
a la temperatura y cargas.

Entre estos encontramos los siguientes.

\begin{description}
  \item[Ensayo de fragilidad Fraass] 
  \item[Ensayo de penetración] se realiza a 25ºC, y nos sirve para medir la
    consistencia del asfalto.
  \item[Ensayo de punto de ablandamiento] nos permite determinar la temperatura
    de ablandamiento.
  \item[Ensayo de viscosidad] 
\end{description}

Otro índice que puede ser interesante es el \textit{indice de penetración de
Pfeiffer y Van Doormaal}, que indica la susceptibilidad térmica del asfalto.
Dependiendo del valor que toma puede ser interpretado el tipo de desempeño
que tendrá, donde un valor entre -1 y 1 es un desempeño moderado.

\subsection{Ensayos de durabilidad}

Se ensaya la capacidad que presenta el asfalto de mantener las propiedades
cohesivas y cementantes con el paso del tiempo, y ciertos agentes tales como
agua, rayos ultravioleta, oxígeno, etc. 

El proceso de mezclado y compactación, a pesar de su corta duración, es crítico.
Es importante controlar las temperaturas y la película de asfalto, que puede
provocar la pérdida de volátiles.

Algunos ensayos son los de \textbf{ensayo de película delgada}.

\subsection{Base asfáltica}

Esta puede ser conformada por asfalto más otro aditivo, tales como:

\begin{itemize}
  \item Con solvente, para conformar asfalto cortado.
  \item Con agua más emulsificante, para formar emulsión asfáltica.
  \item Destilación directa, para conformar cemento asfáltico.
\end{itemize}




\end{document}
