\documentclass[../main.tex]{subfiles}

\begin{document}

\section{Estabilizado granular}

\href{https://youtu.be/1-kH3FTW_-Q}{Clase 20210513}

Un estabilizado granular es una mezcla bien graduada de gravas, arena y finos,
donde los dos primeros se conforma un esqueleto granular firme, y el fino aporta
cierta plasticidad, ya que permiten el llenado de os vacíos del esqueleto 
granular. Así se puede lograr una buena resistencia mecánica.
Esto se fundamenta en la siguiente ecuación

\begin{align*}
  \tau = c + \sigma' * tg(\phi)
.\end{align*}

Se consideran:

\begin{description}
  \item[Fracción gruesa] tiene un tamaño máximo de 1" o 2", y queda retenido en
    el tamiz Nº10.
  \item[Fracción intermedia] si es arena gruesa, se encuentra entre los tamices
    Nº10 y Nº40. Si es arena mediana o fina, entre Nº40 a Nº200.
  \item[Fracción fina] son limos y arcillas, pasante de Nº200.
\end{description}

Su capacidad de carga se estima en un Valor Soporte relativo tal que:
\begin{align*}
  \text{Para una subase} & \hspace{0.25cm} \xrightarrow{\hspace*{0.5cm}} \hspace{0.1cm} \text{VSR} \geq 40\% \\
  \text{Para una base} & \hspace{0.25cm} \xrightarrow{\hspace*{0.5cm}} \hspace{0.1cm} \text{VSR} \geq 80\%
.\end{align*}

\subsection{Los materiales}

Se debe corroborar la idoneidad de los agregados. Se ve:

\begin{description}
  \item[Árido grueso] es preferentemente triturado, reconociendo su forma (lajosidad)
    y su resistencia a la fragmentación (ensayo Los Ángeles).
  \item[Árido intermedio] se análiza su limpieza (esayo Equivalente de arena).
\end{description}

A la misma mezcla se le debe reconocer su permeabilidad y realizarse ensayos de
durabilidad. Esto es particularmente importante, por ejemplo, en capas drenantes.

\subsection{La doficicación}

\subsubsection{La granulometría}

Debe ser tal que asegure la altra fricción interna. La composición se realiza
por tanteos, hasta obtener una mezcla dentro de la banda que se sugiere.
Una vez definida la mezcla, se trata como un suelo.

\subsubsection{La resistencia mecánica}

Lo primero es realizar el ensayo Proctor (encontrar $H_{op}$ y $\gamma_{max}$),
y el Ensayo Valor Soporte Relativo. Si el VSR es menor al requerido, es necesario
realizar una nueva dosificación granulométrica.

\subsubsection{Construcción}

El mezclado de los materiales se realiza preferentemente en planta. Su distribución
es en el lugar, y se realiza con distribuidores mecánicos autopropulsados. Luego
se compacta y se le hace un riego de imprimación.

Este tipo de solución tiene una muy buena respuesta, que además permite ser 
complementado con otro estabilizado, con cemento por ejemplo. 

El mismo puede ser realizar en dos capas sucesivas del paquete estrutural, pero
con distinto grado de compactación.

\section{Suelo cemento}

El suelo cemento es una \textbf{estabilización química}, que resulta de la mezcla
de suelo, cemento portland y agua en cantidades determinadas, que se compacta
a alta densidad y curada.

Las principales propiedades que aporta son resistencia y durabilidad. Químicamente,
su acción es diferente de acuerdo al tipo de suelo, según:

\begin{description}
  \item[En suelos finos] el agua y el cemento forman un hidrato de calcio,
    liberando iones de calcio, así, toma agua de la arcilla, disminuyendo la
    plasticidad y aumentando la resistencia y la durabilidad.
  \item[En limos, arenas y gravas] en estos suelos, el cemento reacciona con los
    elementos silicosos de los suelos, actuando como ligante.
\end{description}

\subsection{Los materiales}

En general, se puede adoptar cualquier tipo de suelo, excepto en orgánicos o 
arcillas muy plásticas. Es partícularmente apto para los suelos granulares, ya
que: se pueden desmenuzar y mezclar fácilmente, y requieren menos cemento. La
mezcla típica es:

\begin{align*}
  \text{Pasante Nº200} & \hspace{0.25cm} \xrightarrow{\hspace*{0.5cm}} \hspace{0.1cm} \text{entre 5\% y 35\%} \\
  \text{Pasante Nº4} & \hspace{0.25cm} \xrightarrow{\hspace*{0.5cm}} \hspace{0.1cm} \geq 55\% 
.\end{align*}

En cuanto el cemento, es generalmente el tipo Portland normal. Dependiendo el 
tipo de suelo variará su tenor en peso. Para suelos A-1-a normalmente es al
rededor de 5\%, mientras que para un A-7 va del 11\% al 16\%.

\subsection{Dosificación}

Se deben alcanzar las exigencias con respecto a la compresión simple y al
desgaste.

El suelo cemento se contrae al endurecer, lo que lo lleva a \textbf{fisurarse}.
Esto depende del tenor de cemento, tipo de suelos, contenido de agua, grado de 
compactación y las condiciones de curado. 

Para disminuir la fisuración hay qeu tener especial cuidado en el curado, y se
recomienda trabajar con una humedad superior a la óptima.

\subsection{Construcción}

Se hace mediante el mezclado de los materiales, que se realiza en planta. 
Normalmente se utiliza una planta con un silo con suelo, un silo con cemento, un
tanque del agua y un tornillo o equipo similar para el mezclado. En el
tornillo se descarga el suelo y se agrega el agua necesaria con el cemento.

Luego, en el lugar se lo descarga, compacta y perfila.

El suelo cemento puede alcanzar resistencias considerables, y generalmente puede
ir por debajo inmediatamente por debajo de las capas de rodamiento.

Su resistencia a la erosibilidad lo hace especialmente apto para colocar por 
debajo de las capas asfálticas o de hormigón.


\end{document}
