\documentclass[../main.tex]{subfiles}

\begin{document}

\subsection{Estabilizado: granular}

\subsection{Bases y sub-bases}

Generalmente esta parte del paquete se conforma de una fracción gruesa, es
decir, retenida en el tamiz Nº10, que es el esqueleto resistente al impacto
y desgaste, y una fracción intermedia, que se compone del pasante del tamiz
Nº40 y Nº200, que sirve para el acuñamiento sin separación de las partículas
grandes.

\subsubsection{Idoneidad de uso}

La composición mineralógica de los agregados determina la buena medida de sus
características. Por lo tanto, al seleccionar una fuente de materiales, se
debe conocer el tipo de roca que es, para determinar sus características.

\subsubsection{Propiedades generales}

Lo primero que se debe distinguir es la \textbf{estabilidad y densidad}, que
se requiere que los materiales granulares tengan una adecuada trabazón 
mecánica, pudiendo soportar los esfuerzos impuestos. Esto se logra mediante
una alta fricción interna, permitiendo una poca deformación bajo cargas.

También se debe evaluar la \textbf{durabilidad}, donde es la resistencia de las
partículas a cambios mineralógicos y desintegración física a causa de los ciclos
de humedecimiento.o

La \textbf{permeabilidad} de un material granular dependerá de su granulometría,
tipo de agregado, tipo de ligante y densidad.

\href{https://youtu.be/1-kH3FTW_-Q}{Clase 20210513}

Un estabilizado granular es una mezcla bien graduada de gravas, arena y finos,
donde los dos primeros conforman un esqueleto granular firme, y el fino aporta
cierta plasticidad, ya que permiten el llenado de os vacíos del esqueleto 
granular. Así se puede lograr una buena resistencia mecánica.
Esto se fundamenta en la siguiente ecuación

\begin{align*}
  \tau = c + \sigma' * tg(\phi)
.\end{align*}

Se consideran:

\begin{description}
  \item[Fracción gruesa] tiene un tamaño máximo de 1" o 2", y queda retenido en
    el tamiz Nº10.
  \item[Fracción intermedia] si es arena gruesa, se encuentra entre los tamices
    Nº10 y Nº40. Si es arena mediana o fina, entre Nº40 a Nº200.
  \item[Fracción fina] son limos y arcillas, pasante de Nº200.
\end{description}

Su capacidad de carga se estima en un Valor Soporte relativo tal que:
\begin{align*}
  \text{Para una sub-base} & \hspace{0.25cm} \xrightarrow{\hspace*{0.5cm}} \hspace{0.1cm} \text{VSR} \geq 40\% \\
  \text{Para una base} & \hspace{0.25cm} \xrightarrow{\hspace*{0.5cm}} \hspace{0.1cm} \text{VSR} \geq 80\%
.\end{align*}

\subsection{Los materiales}

Se debe corroborar la idoneidad de los agregados. Se ve:

\begin{description}
  \item[Árido grueso] es preferentemente triturado, reconociendo su forma 
    (lajosidad) y su resistencia a la fragmentación (ensayo Los Ángeles).
  \item[Árido intermedio] se analiza su limpieza (ensayo Equivalente de arena).
\end{description}

A la misma mezcla se le debe reconocer su permeabilidad y realizarse ensayos de
durabilidad. Esto es particularmente importante, por ejemplo, en capas drenantes.

\subsubsection{Modulo resiliente}

Es un estimativo del modulo de elasticidad que se basa en la relación de 
esfuerzos y deformaciones de cargas rápidas como las que reciben los pavimentos
a través de los vehículos.

Este modulo no es una medida de resistencia de los materiales, ya que no se
ensaya a la rotura.

\subsection{La dosificación}

\subsubsection{La granulometría}

Debe ser tal que asegure la fricción interna. La composición se realiza
por tanteos, hasta obtener una mezcla dentro de la banda que se sugiere.
Una vez definida la mezcla, se trata como un suelo.

\subsubsection{La resistencia mecánica}

Lo primero es realizar el ensayo Proctor (encontrar $H_{op}$ y $\gamma_{max}$),
y el Ensayo Valor Soporte Relativo. Si el VSR es menor al requerido, es 
necesario realizar una nueva dosificación granulométrica.

\subsubsection{Construcción}

El mezclado de los materiales se realiza preferentemente en planta. 
Su distribución es en el lugar, y se realiza con distribuidores mecánicos
autopropulsados. Luego se compacta y se le hace un riego de imprimación.

Este tipo de solución tiene una muy buena respuesta, que además permite ser 
complementado con otro estabilizado, con cemento por ejemplo. 

El mismo puede ser realizar en dos capas sucesivas del paquete estructural,
pero con distinto grado de compactación.

\section{Estabilizado: drenaje}

\subsection{Generalidades}

Generalmente se consideran la infiltración y las aguas subterráneas. 
Podemos distinguir distintos tipos de drenes, tales como: drenaje longitudinal,
drenes transversales y horizontales, bases permeables y sistemas de pozos.

Para reducir el daño, también se pueden tomar otras alternativas, que se 
desarrollan en la diapositiva 7 del documento "06\_Drenaje".

\begin{description}
  \item[Base permeable] capa drenante de granulometría abierta. Esta puede
    ser tratada o no, y puede también ser drenada por los bordes.
    Normalmente se la usa en conjunto con una \textit{capa separadora}, que
    es una capa de granulometría cerrada, y mantiene una separación entre
    la base permeable y la subrasante, dirigiendo la infiltración a los
    drenes horizontales.
  \item[Drenajes longitudinales] generalmente se ejecutan mediante tubería 
    circular, y corre a lo largo de un pavimento, interceptando el agua
    que sale de la estructura. Se ubica normalmente en la cuneta.
  \item[Drenajes transversales] son tuberías cortas, conectadas a los drenajes
    longitudinales, que transportan el agua desde los drenajes longitudinales a
    las cunetas o quebradas.
\end{description}

Un sistema de drenaje típico se puede conformar mediante una base permeable,
una capa de separación, y drenes longitudinales y transversales.

De no utilizarse un sistema de este tipo, se puede dar un \textbf{proceso de}
\textbf{deterioro}, donde se puede producir una migración de suelos finos dentro de la
a capa granular.


\subsection{Método de diseño}

Lo más normal es toma como parámetro de diseño el \textbf{tiempo de drenaje}.
Durante una lluvia, el agua se infiltra en la base permeable hasta que la misma
se satura, y el tiempo de drenaje será el tiempo requerido para drenar una 
cierta cantidad de agua de la capa drenante una vez que la lluvia ha parado.

\section{Estabilizado: cal}

Lo primero es distinguir entre los tipos de cal. Podemos distinguir tanto la 
\textbf{cal aérea} y la \textbf{cal hidráulica}. El endurecimiento se da 
debido a un intercambio de cationes con la arena, donde se pueden producir
a corto plazo la floculación y aglomeración, y a largo plazo el efecto 
puzolánico y la carbonatación.

En el \textbf{corto plazo}, se que genera una reducción contenido de agua,
índice plástico y un aumento de la plasticidad. Por otro lado, en el
\textbf{largo plazo}, se da un efecto puzzolánico y carbonatación.


\subsection{Cal útil vial}

Es el porcentaje de la masa de cal liberado como óxido de calcio, que con agua
es capaz de transformarse en solución de hidróxido de calcio, la sustancia
activa que ataca a los silicatos y aluminatos.

\subsection{Curado}

El curado se da para evitar la evaporación del agua contenida en la masa
de suelo cal y se hace inmediatamente después de terminada la capa, aplicando
un riego de material bituminoso.

\subsection{Aplicaciones de la cal}

\begin{description}
  \item[Mejoramiento de suelos] se utiliza para mejorar la plataforma de 
    trabajo por secado de las arcillas, disminuyendo la sensibilidad al agua.
    Se utiliza para dotar a la subrasante de mayor capacidad portante.
  \item[Estabilización de suelos]  se busca la reacción puzzolánica con suelos
    que contengan arcilla para obtener incrementos de resistencia a compresión
    y usar la mezcla como sub-base.
  \item[Tratado con cal] se refiere a la incorporación de can en el suelo
    de la subrasante con el objeto de aprovechas los efectos a corto plazo
    del uso de la cal.
\end{description}

\subsubsection{Dosificación en mejoramientos de suelos}

Para este tipo de tratamiento se considera hasta un 4\% del peso del suelo,
luego se determina LL, LP y IP, se moldean probetas para ensayar al CBR, y 
se evalúa la resistencia a la compresión a los siete días. Normalmente se tiene
un tenor de CUV de 2\% a 5\%.

\subsubsection{Dosificación en estabilización de suelos}

Se verifica la evolución en el tiempo de la resistencia, y se tiene hasta el
doble de tenor de CUV.

\subsection{Proceso constructivo}

% COMPLETAR

\section{Estabilizado: cemento}

El suelo cemento es una \textbf{estabilización química}, que resulta de la mezcla
de suelo, cemento Portland y agua en cantidades determinadas, que se compacta
a alta densidad y curada.

Las principales propiedades que aporta son resistencia y durabilidad. 
Químicamente, su acción es diferente de acuerdo al tipo de suelo, según:

\begin{description}
  \item[En suelos finos] el agua y el cemento forman un hidrato de calcio,
    liberando iones de calcio, así, toma agua de la arcilla, disminuyendo la
    plasticidad y aumentando la resistencia y la durabilidad.
  \item[En limos, arenas y gravas] en estos suelos, el cemento reacciona con los
    elementos silicosos de los suelos, actuando como ligante.
\end{description}

Este tipo de tratamiento permite aumentar la resistencia a la compresión 
y flexión simple del suelo.

\subsection{Contracción}

Es el resultado de la pérdida de agua por secado y de las reacciones ocurridas
durante la hidratación del cemento. Existen varios factores que influyen en el
mismo tales como:

\begin{itemize}
  \item Tipo y cantidad de cemento.
  \item Contenido de agua aplicado en el campo.
  \item Propiedades de los agregados.
  \item Procedimientos de curado realizados y condiciones de clima.
  \item Rozamiento entre la capa de suelo-cemento y subyacente.
\end{itemize}

\subsection{Resistencia al desgaste}

Es excelente para soportar esfuerzos perpendiculares a la superficie pero 
deficiente para esfuerzos abrasivos. Por esto generalmente se cubre con una capa
de rodamiento de hormigón o asfalto.

\subsection{Los materiales}

En general, se puede adoptar cualquier tipo de suelo, excepto en orgánicos o 
arcillas muy plásticas. Es particularmente apto para los suelos granulares, ya
que: se pueden desmenuzar y mezclar fácilmente, y requieren menos cemento. La
mezcla típica es:

\begin{align*}
  \text{Pasante Nº200} & \hspace{0.25cm} \xrightarrow{\hspace*{0.5cm}} \hspace{0.1cm} \text{entre 5\% y 35\%} \\
  \text{Pasante Nº4} & \hspace{0.25cm} \xrightarrow{\hspace*{0.5cm}} \hspace{0.1cm} \geq 55\% 
.\end{align*}

En cuanto el cemento, es generalmente el tipo Portland normal. Dependiendo el 
tipo de suelo variará su tenor en peso. Para suelos A-1-a normalmente es al
rededor de 5\%, mientras que para un A-7 va del 11\% al 16\%.

\subsection{Fisuración}

Se deben alcanzar las exigencias con respecto a la compresión simple y al
desgaste.

El suelo cemento se contrae al endurecer, lo que lo lleva a \textbf{fisurarse}.
Esto depende del tenor de cemento, tipo de suelos, contenido de agua, grado de 
compactación y las condiciones de curado. 

Para disminuir la fisuración hay qeu tener especial cuidado en el curado, y se
recomienda trabajar con una humedad superior a la óptima.

\subsection{Dosificación}

Para la dosificación se puede adoptar la Norma VN-E20-66 de la DNV. El proceso
consiste en lo siguiente:

\begin{enumerate}
  \item Clasificación del suelo.
  \item Adopción de un tenor de cemento tentativo.
  \item Determinación de $D_{max}$ y $H_{opt}$ mediante Ensayo Proctor.
  \item Determinar resistencia mediante ensayos de compresión y durabilidad.
  \item Adopción de la cantidad óptima.
\end{enumerate}

Donde los tres primeros pasos deben ser iterados con distintas cantidades de 
cemento, a fin de determinar la cantidad óptima.

\subsection{Construcción}

Se hace mediante el \textbf{mezclado} de los materiales, que se realiza en 
planta. Normalmente se utiliza una planta con un silo con suelo, un silo con 
cemento, un tanque del agua y un tornillo o equipo similar para el mezclado. 
En el tornillo se descarga el suelo y se agrega el agua necesaria con el 
cemento.

Luego, en el lugar se lo descarga y distribuye, para luego verificar condiciones
de humedad y granulometría.

Luego se lo puede \textbf{compactar} y perfilar. Para esto se utilizan equipos 
adecuados, tales como compactadores neumáticos. Esto debe hacerse dentro del
plazo de manejabilidad, que normalmente es de dos horas. Este proceso se 
finaliza con el \textbf{sellado final}.

Por último, se da el \textbf{curado} del suelo cemento. Esto se hace con un
riego de curado, que normalmente es con materiales bituminosos, destinados a
mantener la humedad de la capa cementada.

El suelo cemento puede alcanzar resistencias considerables, y generalmente puede
ir por debajo inmediatamente por debajo de las capas de rodamiento.

Su resistencia a la erosibilidad lo hace especialmente apto para colocar por 
debajo de las capas asfálticas o de hormigón.

\section{Estabilización: asfaltos}

Esto se hace temperatura ambiente, por lo que se deben utilizar asfaltos tales
como:

\begin{description}
  \item[Emulsiones asfálticas] es una dispersión homogénea de pequeños glóbulos
    de cemento asfáltico. La emulsión puede ser aniónica o catiónica.
  \item[Asfalto espumado] se forma por inyección de agua fría y aire comprimido
    en cemento asfáltico, haciendo que queden atrapadas burbujas dentro del
    asfalto. Sus caracteristicas más importantes son la relación de expansión
    y la vida media.
\end{description}

\subsection{Mecanismo de estabilización: suelos finos}

En este tipo de suelos, la estabilización consiste en la 
\textbf{impermeabilización}. Como el suelo poseerá cohesión, la función del 
asfalto será la de formar una membrana que impide la penetración del agua,
previniendo cambios de volumen del suelo y reducciones en su resistencia.

\subsection{Mecanismo de estabilización: suelos granulares}

En los materiales granulares, se dan dos mecanismos. Primero el de 
\textbf{impermeabilización}, ya descripto. Además, se da uno de 
\textbf{adhesión} que brinda al agregado la cohesión de la cual carece, 
aumentando la resistencia al corte y a la flexión.

\subsection{Factores}

Algunos factores son: tipo de estabilizarte, tipo y gradación del suelo,
densidad de la mezcla compactada, curado, temperatura de ejecución de los
ensayos y velocidad de aplicación.

\subsection{Emulsión asfáltica}

Para suelos finos se deben cumplir ciertos requisitos, dependiendo de la
plasticidad y la cantidad de material que pasa el tamiz Nº200. Un exceso
de partículas finas se traduce en una superficie específica muy grande, que 
exigiría una proporción de asfalto muy alta.

\subsubsection{Diseño de la mezcla}

Se utiliza la durabilidad como criterio del comportamiento de la mezcla después
de compactada y curada. Existen varios métodos, según el tipo de suelos,
osea de grano fino o granulares.

En el \textbf{método de inmersión-compresión}, se hace lo siguiente:

\begin{enumerate}
  \item Determinación de la humedad óptima de compactación.
  \item Determinación de contenido óptimo teórico de ligante.
  \item Elaboración de mezclas.
  \item Compactación de probetas.
  \item Curado de probetas.
  \item Ensayo de compresión.
  \item Determinación del contenido óptimo de emulsión.
\end{enumerate}

\subsubsection{Módulo dinámico}

Es importante saber que el módulo dinámico de las capas estabilizadas con
emulsión asfáltica tiende a reducirse con el tiempo, a causa de la fatiga por
la aplicación de las cargas de tránsito.

\subsubsection{Comportamiento a la fatiga}

El comportamiento a fatiga de las estabilizaciones con emulsión asfáltica es 
similar al de las mezclas bituminosas.

\subsubsection{Dosificación}

\begin{description}
  \item[Tarea A] Encontrar proporción de suelo y arena según densidad Proctor.
  \item[Tarea B] Encontrar mejor mezcla con emulsión.
  \item[Tarea C] Evaluar eficiencia de la estabilización por medio de absorción.
\end{description}

Se explica más en detalle en la diapositiva 40 de "10 Bases con asfalto".

\subsubsection{Método constructivo}

Puede hacer tanto en el camino, o en planta fija. Luego de aireada la mezcla
 se debe dar un proceso de compactación, con rodillos pata de cabra y neumático,
 y por último se da un perfilado e imprimado.

\subsection{Asfalto espumado}

Suele utilizarse para las siguientes zonas de la gráfica de granulometría

\begin{description}
  \item[Zona A] el material es adecuado para la estabilización de vías de 
    tránsito pesado.
  \item[Zona B] el material es apropiado para estabilización en vías de tránsito
    liviano.
  \item[Zona C] el material es deficiente en finos y no responde bien al 
    tratamiento.
\end{description}

\subsubsection{Diseño de la mezcla}

\begin{enumerate}
  \item Optimización de las propiedades del asfalto espumado.
  \item Determinación del contenido óptimo de humedad.
  \item Elaboración de mezclas de ensayo.
  \item Compactación de probetas de ensayo.
  \item Curado de probetas.
  \item Medida de dimensiones y pesos de probetas.
  \item Ensayo de tracción indirecta.

\end{enumerate}

\subsection{Combinación de estabilizantes}

Consiste en la combinación de estabilizantes es realizar un tratamiento previo
del suelo para modificar algunas características antes de aplicar el 
estabilizarte dominante.

\subsubsection{Combinación cal-cemento}

El cemento no se puede mezclar exitosamente con finos muy plásticos. Al realizar
este tratamiento, se logra que la cal, que se agrega primero, flocule los finos
con una reacción rápida, disminuyendo la plasticidad y reduciendo la humedad, 
y luego el cemento produce una rápida crecida en la resistencia mecánica del 
suelo.

\subsubsection{Combinación cal/cemento-asfalto}

El curado es un factor importante en los tratamientos con asfalto, por lo que
el tratamiento previo del suelo con cal o cemento hace que la estabilización
con el producto asfáltico sea más resistente a la humedad.

\subsubsection{Otros}

También se pueden ver bases permeables o bases de concreto pobre (ver PDF).








\end{document}
