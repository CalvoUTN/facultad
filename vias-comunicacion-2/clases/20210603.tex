\documentclass[../main.tex]{subfiles}

\begin{document}

\subsection{Método AASHTO}

El método puede ser pensado como un equilibrio, donde por un lado se consideran
todos los requerimientos a tener en cuenta, tales como la subrasante, tránsito,
clima y serviciabilidad. En cuanto al tránsito, el método considera un
$N_{80KN}$, que será aproximadamente $2.26 N_{10t}$ del método Shell. En
cuanto el clima, se tiene en cuenta un factor regional $FR$.

Por el otro lado, se requieren tener en cuenta ciertos indices, tales como
coeficientes estructurales por unidad de espesor $a_i$ y espesores $e_i$. Esto
permite diseñar una carpeta que pueda cubrir las resistencias requeridas.

\section{Pavimentos asfálticos}

\subsection{El petróleo}

Es un componente elemental del asfalto, que se compone de de carbono, hidrógeno,
azufre y nitrógeno.

\subsection{Ligantes hidrocarbonados}

Se pueden encontrar ligantes bituminosos, que son derivados del petróleo, tales
como los betunes. Los mismos pueden ser betunes fluidificados, emulsiones
bituminosas y ligantes bituminosos modificados.

Tienen como caracteristicas lo siguiente:

\begin{enumerate}
  \item Mezclas complejas de hidrocarburos.
  \item Color negro o muy oscuro.
  \item Materiales aglomerantes (adhesividad).
  \item Carácter termo-plástico (Susceptibilidad térmica).
  \item Pérdida de propiedades iniciales (envejecimiento).
\end{enumerate}

\subsubsection{Estructura coloidal}

El asfalto cuenta con:

\begin{description}
  \item[Asfáltenos] son partículas sólidas bituminosas, que son de alta
    viscosidad, y proveen tanto elasticidad, resistencia y cohesión
  \item[Maltenos] estos pueden dividirse entre resinas y aceites. Los primeros
    son muy sensibles a la temperatura, y son responsables de la 
    viscoelasticidad. Se pueden transformar en asfáltenos mediante la oxidación.
    Los aceites aumentan la fluidez, y en la oxidación se transforman en 
    resinas y asfáltenos.
\end{description}




\end{document}
