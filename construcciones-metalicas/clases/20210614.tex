\documentclass[../main.tex]{subfiles}

\begin{document}

\section{Resistencia de diseño al corte}

Se deben considerar la plastificación del alma, pandeo inelástico del alma y
pandeo elástico del alma.

Generalmente se verifica como:

\begin{align*}
  \phi_v i &* V_n  \\[5pt]
  V_n = \tau_{cr} * A_w &= (0.6*F_{yw} * C_v)*A_w
.\end{align*}

Donde el valor del $\tau_{cr}$ se puede obtener a partir de la formula de
Juraski como:

\begin{align*}
  \tau = \frac{V*S}{b*J} 
.\end{align*}

Donde $S$ es el momento estático de la sección con respecto al baricentro.

En el caso de \textbf{vigas armadas} se debe tener en cuenta un efecto de campo
de tensión diagonal. El alma atiesada por las alas tiene una resistencia post
pandeo, incrementando la capacidad del corte.

\end{document}
