
\begin{document}

\section{Generalidades. La seguridad estructural.}

\subsection{¿Qué es el acero?}

Son aleaciones de hierro y carbono, que pueden ser definidos como 
\textbf{aceros de bajo carbono} y \textbf{aceros de alto carbono}, que varian 
mucho en sus propiedades.

Ademas puede contener otros elementos como magnesio, cobre, silicio, fosforo y
azufre. Cada una de estas proporciona un efecto distinto, que modifica las 
caracteristicas del acero.

En general, para la compra del acero solo es necesario aclarar la tensión de 
fluencia del mismo. Es también importante tener en cuenta dos caracteristicas
del acero:

\begin{itemize}
  \item \textbf{Tenacidad:} capacidad de almacenar energía antes de la rotura.
  \item \textbf{Ductilidad:} capacidad de deformarse elásticamente.
\end{itemize}

\subsection{Fabricación}

\subsubsection{Proceso siderúrgico}

Es el proceso para obtener el acero, desde la llegada del hierro hasta el producto
final. Se pueden considerar dos alternativas: \textit{alto horno seguido del convertidor}
\textit{al oxígeno}, o un \textit{horno de reducción directa seguido del horno eléctrico}
\textit{de arco.} 

En general, el proceso comprende la eliminación progresiva de las impurezas del
hierro. Se consideran tres etapas:

\begin{itemize}
  \item Preparación de las materias primas.
  \item Reducción.
  \item Acelereación.
\end{itemize}

\subsubsection{Taller metalúrgico}

Una vez finalizado el proyecto de estructura, la misma debe ser fabricada en un
\textbf{taller metlaúrgico}. Es muy importante la facilidad de fabricación de 
la pieza. Se deben estudiar:
\begin{itemize}
  \item Posición de agujeros y soldaduras, para evitar excesivo manipuleo.
  \item Tolerancias exigibles en el proyecto.
  \item Dimensiones a fabricar.
  \item Proceso de deformación de barras con ejes no recto.
\end{itemize}

También es importante la estandarización de elementos en la estructura.

\subsection{Propiedades físicas y químicas del acero}

\begin{itemize}
  \item Buena ductilidad y maleabilidad.
  \item Conductividad térmica elevada.
  \item Conductividad eléctrica elevada.
  \item Brillo metálico.
  \item Resistencia a la corrosión.
\end{itemize}

Además, es un material isotropo, es decir que su resistencia a la tracción es 
idéntica a la resistencia a la compresión.

\subsubsection{Efectos de la temperatura}

Podemos ver que a medida aumenta la temperatura a la que este trabajando el acero,
disminuirá la resistencia de la misma. Es tanto asi que a 600ºC el acero tiene
menos que la mitad de su resistencia y pierda toda linealidad.

Este punto es una ventaja de la madera con respecto al acero, ya que hasta que no
este quemada, mantiene su resistencia.

\end{document}
