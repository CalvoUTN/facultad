\documentclass[../main.tex]{subfiles}

\begin{document}

\section{Flexión}

Un miembro sometido a flexión es un miembro estructural sobre el que actuan
cargas perpendiculares a su eje que producen flexión y corte. 

Normalmente se utilizan perfiles canal, vigas W, secciones armadas y abiertas.

Podemos distinguir dos tipos de vigas según su soporte lateral:

\begin{description}
  \item[Vigas con soporte lateral adecuado] son aquellas con arriostramiento poco
    espaciado, y donde la inestabilidad global no controla capacidad.
  \item[Vigas sin soporte lateral] son arriostramientos con espaciado mayor, y
    donde la inestabilidad global puede controlar la capacidad.
\end{description}

De acuerdo a la geometría de la sección

\begin{description}
  \item[Vigas de sección compacta] 
  \item[Vigas de seción no compacta] 
  \item[Vigas de sección esbelta] 
\end{description}

\subsection{Modos de falla}

Se pueden distinguir varios tipos de falla, tales como:

\begin{description}
  \item[Plastificación de la sección] se da cuand se puede llegar a la
    plastificación total sin necesitar de llegar a las tensiones críticas de
    pandeo.
  \item[Pandeo lateral torsional] 
  \item[Pandeo local] se puede dar tanto en el alma o en el ala.
\end{description}

Las hipótesis iniciales son las comunes: no hay inestabilidad, comportamiento
perfectamente elástico-plástico, no hay fractura ni fatiga. Se Puede conseguir
un momento plástico $M_p$ que es donde se tiene la máxima tensión debido a que 
la sección se encuentra plastificada.

A partir del desarrollo de la sección sometida a un momento plástico se puede 
hallar el \textbf{módulo plástico} que se define como:

\begin{align*}
  Z_x = A_c * \overline{y_c} + A_t * \overline{y_t}
.\end{align*}

Es te valor se puede relacionar con el modulo elástico como:

\begin{align*}
  \alpha = \frac{Z_x}{S_x}
.\end{align*}

Esta relación tiende a ser 1.12 a 1.2 para secciones doble T.

Cuando consideramos el \textbf{pandeo lateral torcional}, podemos hablar de sus
causas, que son tensiones residuales e imperfecciones iniciales.
Mediante el uso de elementos finitos se puede desarrollar la siguiente formula
para determinar el momento crítico:

\begin{align*}
  M_{cr} = \frac{\pi}{L} * \sqrt{E*I_y *GJ} *\sqrt{1+\frac{\pi^2}{L^2}*\frac{E*C_w}{GJ} }  
.\end{align*}

Vemos que los factores que afectan al momento crítico son princialmente la
distancia de arriostriamiento, donde a mayor longitud, decrese el momento.

\subsubsection{Longitudes de arriostramiento límite}

La longitud no arriostrada límite se determina de la siguietne forma:

\textit{Para cargas aplicadas enyy el alma o en el ala inferior de la viga}

\begin{align*}
  L_p = 1.76 * r_y \sqrt{\frac{E}{F_{yf}}} 
.\end{align*}

\textit{Para cargas aplicadas en el ala superior de la viga}

\begin{align*}
  L_p = 1.59 * r_y \sqrt{\frac{E}{F_{yf}}} 
.\end{align*}

Por otro lado, se determina la longitud lateralmente no arriostrada $L_r$ y
su correspondiente momento $M_r$ con otras formulas que se encuentran en el 
reglamento CIRSOC 201.




\end{document}
