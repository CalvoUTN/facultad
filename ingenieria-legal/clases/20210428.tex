\documentclass[../main.tex]{subfiles}

\begin{document}

\section{Responsabilidad civil: por daños y perjuicios}
\href{https://youtu.be/EdKKZ3BBUnE}{Clase 20210428}

Primero se aclara el significado de \textbf{denunciar}, que es poner de
manifiesto un hecho que \textit{puede} ser delictivo. Se puede, por ejemplo,
denunciar el domicilio.

La responsabilidad social implica, entre ciertas cosas, que un funcionario público
esta obligado a denunciar una actividad delictiva.

En este caso es necesario tener en cuenta dos \textbf{daños por culpa}, que son
de involuntarios.

\vspace{0.333cm}
\begin{center}
    \fbox{\begin{minipage}{0.85\textwidth}
        La responsabilidad es la obligación civil de evitar causar daños a otros;
        y en caso de haberse causado un daño, es la obligación civil de no
        agravarlo y de volver las cosas a su estado anterior, si fuese posible,
        además de reparar o resarcir la pérdida causada, el mal inferido o el
        daño originado.
    \end{minipage}}
\end{center}
\vspace{0.3cm}

La función es primeramente de \textbf{prevenir}, y si no es posible prevenirlo,
\textbf{repararlo}. La primera parte ha sido incorporada en caracter general en 
el nuevo Código Civil y Comercial.

\subsection{Prevención}

Para el Código, la prevención significa tres cosas:

\begin{description}
  \item[Evitar] causar un daño no justificado.
  \item[Adoptar] de buena fe y conforme las circunstancias, las medidas razonables
    para \textbf{evitar} que se produzca un daño, o disminuir su magnitud; si
    tales medidas evitan o disminuyen la magnitud de un daño del cual un tercero
    sería responsable, tiene derecho a que éste le reembolse el valor de los
    gastos en que incurrió, conforme a las reglas del enriquecimiento sin causa.
  \item[No agravar] el daño, si ya se produjo.
\end{description}


\subsection{Reparación}

La violación del deber de no dañar a otro, o el incumplimiento de una
obligación, da lugar a la reparación del daño causado.

Es importante la distinción entre \textit{obligación} y \textit{deber}. La diferencia
es que en la obligación siempre hay una contraparte que puede exigir el 
cumplimiento de la obligación. Por otro lado, un deber también es un mandato,
pero por el otro lado no tenemos una persona que puede exigir el cumplimiento,
sino que es la comunidad o Estado que espera dicha conducta.

\subsection{Clasificación de responsabilidad}

En general, podemos distinguirlas según la causa fuente que origina la obligación,
o el factor de atribución de responsabilidad.

\begin{description}
  \item[Según su causa] en este caso podemos distinguir la causa según si son 
    contractuales o extracontractuales. 
    
  \item[Según su atribución] se distinguen entre subjetivas y objetivas, según
    el Código Civil. Lo subjetivo es lo que una persona hace, mientras que 
    la objetiva es por lo que la ley manda.
\end{description}

El \textbf{factor objetivo} cuando la culpa del agente es irrelevante a los
efectos de atribuir responsabilidad. En tales casos, el responsable se libera
demostrando la causa ajena, excepto disposición legal en contrario.

\subsubsection{Factores subjetivos: culpa}

\vspace{0.333cm}
\begin{center}
    \fbox{\begin{minipage}{0.85\textwidth}
        La culpa consiste en la omisión de la diligencia debida según la
        naturaleza de la obligación y las circunstancias de las personas, el
        tiempo y el lugar. Comprende la imprudencia, la negligencia y la
        imperifica en el arte o profesión..."
    \end{minipage}}
\end{center}
\vspace{0.3cm}

\subsubsection{Factores subjetivos: dolo}

\vspace{0.333cm}
\begin{center}
    \fbox{\begin{minipage}{0.85\textwidth}
        "... El dolo se configura por la producción de un daño de manera 
        intencional o con manifiesta indiferencia por los derechos ajenos."
    \end{minipage}}
\end{center}
\vspace{0.3cm}

\end{document}
