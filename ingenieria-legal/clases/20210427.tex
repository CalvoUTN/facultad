\documentclass[../main.tex]{subfiles}

\begin{document}

\href{https://youtu.be/ijOuly7QM9o}{Clase 20210427}

\section{Contratos}

\subsection{Generalidades}

Un contrato es un \textbf{acto jurídico} mediante el cual dos o mnás partes
manifiestan su consentimiento para crear, regular, modificar, transferir o
extinguir relaciones jurídicas patrimoniales. Se encuentra desarrollado en el
Art. 957 del CCC. Un \textbf{contrato} es una ley entre las partes.

Las partes son \textbf{libres} para celebrar contrato y determinar su contenido,
dentro de los límites impuestos por la ley, el orden público, la moral y las
buenas costumbres.

\subsubsection{Efectos vinculantes}

Todo contrato válidamente celebrado es \textbf{obligatorio para las partes}. Su
contenido sólo puede ser modificado o extinguido por acuerdo de partes o en 
los supuestos en que la ley lo prevé.

Lo último es importante, porque le otorga \textbf{facultades a los jueces}, donde
los mismos pueden modificar estipulaciones de los contratos siempre y cuando sea
a pedido de una de las partes cuando lo autoriza la ley, o de oficio cuando se
afecta de modo manifiesto el orden público.

\subsubsection{Buena fe}

Los contratos deben celebrarse, interpretarse y ejecutarse de buena fe, es decir
que obligan no sólo a lo que esta formalmente expresado, sino a todas las
consecuencias que puedan considerarse comprendidas en ellos, con los alcances en
que razonablemente se habría obligado un contratante cuidadoso y previsor.

\subsubsection{Carácter de las normas legales}

Las normas legales relativas a los contratos son suplementarias de la voluntad
de las partes, al menos que por su modo de expresión o contexto, serán indisponibles.
Las normas serán \textit{supletorias} o \textit{indisponibles}. 

\subsubsection{Prelación de normas}

\vspace{0.333cm}

De Art. 934 de CCC.

\begin{center}
    \fbox{\begin{minipage}{0.85\textwidth}
        Cuando concurren disposicion de este Código y de alguna ley especial,
        se aplican con el siguiente orden de prelación:

        \begin{itemize}
          \item Norma indisponsible de la ley especial y de este Código.
          \item Normas partículares del contrato.
          \item Normas supletorias de la ley especial.
          \item Normas supletorias de este Código.
        \end{itemize}
    \end{minipage}}
\end{center}
\vspace{0.3cm}

\subsubsection{Integración del contrato}

Del Art. 964 de CCC.

\vspace{0.333cm}
\begin{center}
    \fbox{\begin{minipage}{0.85\textwidth}
        El contenido del contrato se integra con:

        \begin{enumerate}
          \item Normas indisponibles.
          \item Normas supletorias.
          \item Usos y prácticas.
        \end{enumerate}
    \end{minipage}}
\end{center}
\vspace{0.3cm}

El último punto es bastante complejo, ya que dependerá en gran medida del lugar
de donde se habla. Los \textbf{usos y prácticas} son del lugar de celebración, en 
cuanto sean aplicables porque hayan sido declarados obligatorios por las partes
o porque sean ampliamente conocidos y regularmente observados en el ámbito en 
que se celebra el contrato, excepto que su aplicación sea irrazonable.

\subsection{Clasificación de contratos}

Podemos distinguir tres tipos de contratos:

\begin{description}
  \item[Unilaterales] son contratos unilaterales cuando una parte se obliga hacía
    la otra sin que ésta quede obligada. Normalmente se da en donaciones. Los 
    tipos de obligaciones pueden ser de \textit{dar, hacer y no hacer}.
  \item[Bilaterales] son bilaterales aquellas cuando las partes se obligan
    recíprocamente la una hacía la otra.
  \item[Plurilaterales] es una extensión de la clasificación anterior. 
\end{description}

\subsubsection{Onerosos y gratuitos}

Los contratos son a título \textbf{oneroso} cuando las ventajas que procuran
a una de las partes les son concedidas por una prestación que ella ha hecho
o se obliga a hacer a la otra.

Por otro lado, tenemos a titulo \textbf{gratuito} cuando aseguran a uno o a otro
de los contratantes alguna ventaja, independientemente de toda prestación a su
cargo.

\subsubsection{Conmutativos y aleatorios}

Los contratos a título oneroso son \textbf{conmutativos} cuando las ventajas
de las partes son conocidas cuando se celebra el contrato, a cuando son 
\textbf{aleatorias}, cuando las ventajas o las pérdidas, para uno de ellos o
para todos, dependen de un acontecimiento incierto.

\subsubsection{Formales y no formales}

Son \textbf{no formales} cuando la Ley no establece una forma determinada, y 
\textbf{formales} aquellos para los que la Ley exige una forma determinada. Dentro
de los mismos pueden ser \textit{solemnes} \textit{no solemnes}. 



\end{document}
