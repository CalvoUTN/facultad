\documentclass[../main.tex]{subfiles}

\begin{document}

\section{El Derecho}

\subsection{Las normas}

Se pueden distinguir tres tipos de normas:

\begin{itemize}
  \item \textbf{Normas jurídicas:} este tipo de normas son dictadas por un ente
    competente, son heterónomas (es decir que se le imponen a la persona), 
    bilaterales, coercibles y externas.
  \item \textbf{Normas sociales:} son reglas generales de comportamiento definidas
    por la sociedad. Son heterónomas, unilaterales, externas y no coercibles.
  \item \textbf{Normas religiosas:} son impuestas por una comunidad religiosa, y
    tienen las mismas caracteristicas que una norma social.
  \item \textbf{Normas morales:} son impuestas por la misma persona. Son autónomas,
    unilaterales, incoercibles e internas.
\end{itemize}

\subsection{Derecho público y privado}

Podemos diferenciar entre dos tipos de derecho: el \textbf{derecho público} y el
\textbf{derecho privado}. Se distinguen de la siguiente manera:

\begin{itemize}
  \item \textbf{Derecho público:} es aquel que regula cuestiones relacionadas con
    cargos públicos, sus poderes y facultades democráticas. Se basa en el 
    \textit{principio de la desigualdad}. Se maneja mediante \textit{imperativas}.
  \item \textbf{Derecho privado:} es aquel que regula y ampara los derechos propios
    de la persona en cuanto a su autonomía individual, como la administración del
    propio patrimonio. Se basa en el \textit{principio de igualdad.}
\end{itemize}

Partícularmente podemos encontrar otras \textbf{ramas del derecho}, como:

\begin{enumerate}
  \item Derecho constitucional: es la rama que estudia las leyes fundamentales 
    que definen un Estado. Se puede ver desde una vista formal o una forma material.
  \item Derecho administrativo: regula la organización, funcionamiento y control
    de la administración pública. Aplica a toda actividad temporal o permanente.
  \item Derecho tributario: estudia el ordenamiento jurídico que regula el 
    establecimiento y aplicación de los tributos. Estudia las normas por la que
    el Estado consigue los tributos necesarios para solventar el gasto público.
  \item Derecho penal: regula la potestad punitiva del Estado, asociado a una 
    acción estrictamente determinada por la ley con una sanción o medida de seguridad.
  \item Derecho internacional: estudia las normas que tienen como objetivo 
    contribuir a la relación entre distintos Estados. Pueden existir en el ámbito privado y público.
\end{enumerate}

\section{Supremacía constitucional}

La Constitución Nacional, tratados internacionales y ordenamiento jurídico federal
son jerárquicamente superiores al ordenamiento jurídico provincial. Esto queda 
establecido en el Art.31:

\vspace{0.333cm}
\begin{center}
\fbox{\begin{minipage}{0.85\textwidth}
Ésta Constitución, las leyes de la Nación que en su consecuencia se dicten por el Congreso y los tratados con las potencias extranjeras son la ley suprema de la Nación; y las autoridades de cada provincia están obligadas a conformarse a ella, no obstante cualquiera disposición en contrario que contengan las leyes o constituciones provinciales, salvo para la provincia de Buenos Aires, los tratados ratificados después del Pacto de 11 de noviembre de 1859. 
\end{minipage}}
\end{center}
\vspace{0.3cm}

Además, el gobierno establece para \textbf{tratados internacionales} que esta obligado
a afianzar relaciones de paz y comercio con protencias extranjeras. Estas dos 
cosas significa que las leyes provinicales u otros reglamentos no pueden lesionar
los derechos otorgados en los tratados o la Constitución Nacional. 

\subsection{Medios de tutela}

\subsubsection{Habeas corpus}

Cuando el derecho lesionado, restringido, alterado o amenazado fuera la libertad 
física, o en caso de agravamiento ilegítimo en fomra o condiciones de detención,
o en el de desaparación forzada de personas, la acción de \textbf{hábeas corpus} 
podrá ser interpuesta por el afectado o cualquier y el juez deberá resolver de 
inmediato.

\subsubsection{Amparo individual}

Se usa cuando existe una falta \textit{evidente} e inmediata que necesita una 
respuesta expédita. Existe también el concepto de \textbf{amparo colectivo} y
\textbf{amparo mora}, que tienen fines parecidos. En principio, se define como:

\space{0.3cm}
\begin{center}
\fbox{\begin{minipage}{0.85\textwidth}

Toda persona puede interponer acción expedita y rápida de amparo, siempre que no
exista otro medio judicial más idóneo, contra todo acto u omisión de autoridades
públicas o de partículares, que en forma actual o inminente lesione, restringa,
altere o amenace, con arbitrareidad o ilegalidad manifesta derechos y garantías
reconocidas por esta Constitución, un tratado o una ley.
\end{minipage}}
\end{center}
\vspace{0.3cm}

\subsection{Clases}

% COMPLETAR

\section{La persona y el derecho}

En principio, se pueden distinguir dos tipos de personas: \textbf{peronas humanas}
y \textbf{personas jurídicas}.

\subsection{Persona jurídica}
Son todos los entes a los cuales el ordenamiento jurídico les confiere aptitud para
adquirir derechos y contraer obligaciones para el cumplimiento del objeto de su 
creación. Es importante que la persona jurídica tiene una personalidad \textit{distinta}
a la de sus miembros.

Podemos distinguir dos tipos:

\begin{description}
  \item[Persona jurídica pública] son el Estado nacional y provincial, 
      y otros entes autárquicos. Se rigen en su reconocimiento, comienzo, capacidad,
      funcionamiento y fin de su existencia por las leyes y ordenamientos de la
      Constitución.
  \item[Person jurídica privada] son las sociedades, asociaciones civiles, fundaciones,
    mutuales, consorcios y otas. Que el Estado participe en ellas o no no implica que
    modifique su cáracter priviado, auqnue la leye puede preveer obligaciones
    diferenciadas, considerando el interés público.
\end{description}

\subsection{Atributos de la personalidad}

Son aquellas propiedades o características de identidad propias de la persona 
como titulares de derechos y obligaciones. Estas son inherentes, úncas, inalienables,
imprescriptibles, irrenunciables e inembargables.

Para las personas humanas se establecen los siguientes atributos:

\begin{itemize}
  \item Capacidad.
  \item Nombre.
  \item Domicilio.
  \item Patrimonio.
  \item Estado civil.
\end{itemize}

\subsubsection{El nombre}

En \textbf{personas humanas} corresponde al conjunto de prenombre y apellido, 
mientras que para \textbf{personas jurídicas} corresponde a la Razón social.

La elección del nombre en personas humanas corresponde a los padres o protector,
y se pueden dar hasta tres pronombres no idénticos a lo de los hermanos. 

En cambio, para personas jurídicas el nombre debe identificarlas como tal, y si se
encontrasen en liquidación, su nombre también debe aclarar eso. Aún más, su nombre
debe satisfacer recaudos de veracidad, novedad y aptitud distintiva de otras 
marcas, nombres de fantasía u otras.

\subsubsection{La capacidad}

Podemos encontrar dos tipos de capacidades. Primero tenemos la \textbf{capacidad}
\textbf{de derecho}, que establece que toda persona humana goza de la aptitud para
ser titular de derechos y deberes jurídicos. La ley podría limitar esta capacidad
respecto de hechos, actos simples y otros. 

Por otro lado, tenemos la \textbf{capacidad de ejercicio}, que implica que toda 
persona puede ejercer por sí misma sus derechos, aunque existen excepeciones
expresas tales como para personas por nacer, menores de edad y en caso de que se
tenga una sentencia judicial por insania.

Las \textbf{personas jurídicas} tiene capacidad para ser sujetos de derecho y 
obligaciones, y esta limitado por el alcance de su objeto social, y la ejerción
de esta capacidad se da mediante sus representantes.

\subsubsection{El domicilio}

\begin{description}
  \item[Domicilio real] para \textit{personas humanas} se refiere al lugar de
    residencia habitual. Si contara con una actividad profesional se tiene en el
    lugar donde se desempeña la misma.
  \item[Domicilio legal] es el lugar donde la Ley presume que la persona reside de
    forma permanente para el ejercicio de sus derechos y cumplimiento de sus deberes.
  \item[Domicilio especial] las partes de un contrato pueden elegir un domicilio
    para ejercer los derechos y oblicaciones que de él emanan.
  \item[Domicilio y sede social] son partículares de las personas jurídicas privadas,
    y es el domicilio que se fija en sus estatutos y en la autorización que se le
    dió para funcionar. Esta puede contar con \textbf{sucursales} que son domicilios
    especiales, lugar destinado a la ejecución de las obligaciones allí contraídas.
\end{description}

\subsubsection{El patrimonio}

En las \textbf{personas humanas} es el conjunto de derechos y obligaciones que son susceptibles
a valorarse económicamente. En las \textbf{personas jurídicas} se consideran
también los medios económicos que les permiten realizar sus fines. Se pueden
distinguir como:

\begin{itemize}
  \item Bienes.
  \item Cosas inmuebles.
    \begin{itemize}
      \item Por su naturaleza.
      \item Por su accesión.
    \end{itemize}
  \item Cosas muebles.
      \begin{itemize}
        \item   Que pueden desplazarse por sí mismas.
        \item Que pueden desplazarse por fuerza externa.
      \end{itemize}
\end{itemize}

\subsubsection{El estado civil}

Es un atributo \textbf{exclusivo} de las personas humanas. Consiste en la situación
partícular de la persona con respecto a su familia, sociedad y Estado.

\subsubsection{Nacionalidad}

Si bien no es un atributo, hoy en día prácticamente \textit{todas} las personas
humanas tienen una, y es inherente a la persona.

\subsection{Clases}


\section{Organización del Estado Argentino}

\subsection{Clases}

\section{Hechos y actos jurídicos}

Un \textbf{hecho jurídico} es un fenómeno, suceso o situación que da lugar al
nacimiento, adquisición, modificación, conservación, transmisión o extinción de
derechos u obligaciones. Estos pueden haber sido interveneidos por el ser humano
o no , por lo que se pueden clasificar en \textit{actos naturales} o \textit{actos humanos}.

\subsection{Hechos humanos}

Se pueden distinguir dependiendo de si se da un \textit{ejercicio de la voluntad}
o no. Los \textbf{actos humano} son hechos volutnarios, y pueden ser:

\begin{description}
  \item[Lícitos] no estan prohibidos por la ley. Puede ser un acto jurídico o un
    acto simple.
  \item[Ilícitos] si están prohibidos por la ley. Pueden ser dolo o culpa.
\end{description}

Cuando se habla de \textit{fin inmediato de generar efectos jurídicos}, que aplica
para los actos jurídicos, se refiere a crear, modificar, transferir, conservar,
o extinguir derechos y oblgiaciones.

\subsection{Actos humanos}
La diferencia entre \textbf{actos humanos} y hechos es que el primero tiene como 
finalidad una consecuencia jurídica, pudiendo mostrar una manifestación de crear,
modificar o extinguir un derecho.

Si bien \textit{todo acto jurídico es un hecho jurídico}, solo los hechos jurídicos
que sean de origen humano, lícito y voluntario son actos jurídicos. Podemos, entonces
distinguir los hechos según:

\begin{description}
  \item[Hechos humanos voluntarios] son aquellos que son ejecutados con discernimiento,
    intención y libertad. Pueden ser licitos o ilicitos.
  \item[Hechos humanos involuntarios] si faltase alguno de estos elementos, el 
    autor del hecho carecería de responsabildad, porque su voluntad estaría viciada,
    y el hecho sería considerado \textit{involuntario}.
\end{description}

\subsection{Hechos naturales}

Son aquellos que aconteces sin intervención del hombre, tal como el paso del tiempo,
la lluvia, granizo, etc.

Pueden generar consecuencias jurídicas y por esta razón son denominados \textit{hechos}
\textit{jurídicos.} 

\subsection{Prescripción}

Es un tipo de hecho jurídico que se da por el paso del tiempo.
Se define como \textbf{prescripción} a \textit{una forma de adquirir cosas ajenas}
\textit{o bien de extinguir las acciones y derechos ajenos, por haberse poseído}
\textit{dichas cosas o no haberse ejercido dichas acciones y derechos durante}
\textit{un perídodo de tiempo determinado.}

Podemos ver dos tipos de prescripciones:

\begin{description}
  \item[Adquisitiva] se refiere al modo de adquirir el \textbf{dominio} o los
    derechos reales por la posesión a título de dueño, continuada por el tiempo
    señalado por la Ley.

    Otra palabra para la misma es \textit{usucapión}, que consiste en la posesión
    de un objeto mediante el \textit{animus domini}, que consiste en la capacidad
    de dueño ininterrumpido de forma pacifica.

    En el caso de las propiedades normalmente consiste en 20 años, al menos que
    sea de \textit{buena fe}, donde la prescripción es en 10 años.

  \item[Liberatoria]  es la manera de \textbf{extinguir} las acciones ligadas a 
    derechos de contenido patrimonial por la inactividad del acreedor y por el
    transcurso del tiempo. 

    En este caso lo que se pierde es el derecho a la acción
    de reclamar ante un juez la deuda, pero no es que existe una pérdida de la
    \textbf{deuda moral}, es decir, que si se te paga luego de plazo no es un 
    \textit{regalo}, sino es un pago de la deuda vencida.
\end{description}

\subsubsection{Paralización del computo por plazos de prescripción}

Se pueden dar los siguientes casos:

\begin{description}
  \item[Interrupción] deja sin efecto todo el plazo trasncurrido hasta el momento
    en que se produce el acto interruptivo, exige que comience a contarse nuevamente
    el plazo.

    Un ejemplo, en caso de una prescripción liberatoria se da por una demandas.

  \item[Suspensión] detiene el cómputo del plazo de prescripción durante todo el
    tiempo que dure la situación suspensiva. Una vez desaparecida permite que
    el plazo se integre sumando el tiempo transcurrido con anterioridad a la 
    suspensión.
    
    Un ejemplo de una suspensión es en un acto mediatorio.
\end{description}

\subsection{Elementos esenciales}

Son aquellos elementos que de manera indispensable integran el acto jurídico. Se
distinguen cuatro: el sujeto, el objeto, la causa y la forma.

\subsubsection{El sujeto}

Son las personas que intevienen en el acto jurídico, y pueden hacerlo de forma 
personal o a través de un representante que podrá ser \textit{legal} o \textit{voluntarios}.

\subsubsection{El objeto}

El objeto jurídico \textbf{no debe ser} un hecho imposible o prohibido por la
ley, contrario a la moral, a las buenas costumbres, al orden público o lesivo 
de los derechos ajenos o la dignidad humana.

\subsubsection{La causa}

Es el fin inmediato autorizado por el ordenamiento jurídico que ha sido determinante 
de la voluntad. También integran la causa los motivos exteriorizados cuando sean
lícitos y hayan sido incorporados de forma expresa.

\subsubsection{La forma}
Se pueden dar dos tipos:

\begin{description}
  \item[Libertad de formas] si la ley no designa una forma determinada para la
    exteriorización de la voluntad, las partes pueden utilizar la que estimen
    conveniente. Estas pueden convenir una más exigente que la impuesta por la ley.
  \item[Forma impuesta por la ley]  el acto que no se otorga en la forma exigida
    por la ley no queda concluido como tal mientras no se haya otorgado el 
    instrumento previsto, pero vale como acto en el que las partes se han 
    obligado a cumplir con expresada formalidad, excepto que ella se exija bajo
    sanción de nulidad.
\end{description}

\subsection{Clasificación de actos segun su forma}

Los actos según su forma pueden ser \textbf{formales} cuando su validez y eficiencia
dependerán de la celebración del acto de acuerdo a las formas que determina la
ley. Por otro lado serán \textbf{no formales} para aquellos que la ley \textbf{no}
señala una forma determinada.

\subsubsection{Actos formales}

Además, los actos formales pueden distinguirse como \textbf{actos no solemnes},
que son aquellos en los que la omisión de la forma no determina la nulidad del
acto, pero \textit{impide que se produzcan sus efecto jurídicos}, tales como
la compraventa de un inmueble realizada por boleto y no por escritura pública.

Por otro lado tenemos los \textbf{actos solemnes}, en que la forma exigida es 
un requisito inexcusable de la validez del acto jurídico, tales como el matrimonio.

\subsubsection{Expresión escrita}

Se puede dar por \textit{instrumentos públicos} o por \textit{instrumentos partículares}
\textit{firmados o no}. Puede hacer constar \textbf{cualquier} soporte, siempre
que sea representado con texto inteligible, aunque su lectura exija medios técnicos.

\subsubsection{La firma}

La misma prueba la autoría de la declaración de voluntad expresada en el texto al
cual corresponde. Debe consistir en el nombre del firmante o en un signo.

En caso de instrumentos generados por medios electrónicos, se pude utilizar la
\textit{firma digital}, que asegure indudablemente la autoría e integridad del
instrumento.

\subsubsection{Intrumentos públicos}

Los instrumentos públicos deben cumplir con tres requisitos: 

\begin{enumerate}
  \item Intervención de un funcionario público.
  \item Competencia del funcionario público.
  \item Cumplimiento de formalidades.
\end{enumerate}

Estos hacen fe de que cuando se ha realizado un acto, la fecha, el lugar, y los
hechos que el oficial público enuncia como cumplidos por él o ante él \textit{hasta}
\textit{que se declare falso por juicio civil o criminal}.

Es importante que los \textit{testigos de un instrumento público} y el oficial
público que lo autorizó \textbf{no pueden contradecir su contenido}, si no alegan
que era víctimas de violencia.

\subsubsection{Instrumentos privados}

Son aquellos que presentan estas partícularidades:
\begin{itemize}
  \item Sin intervención de un funcionario.
  \item No sometidos a formalidades.
  \item Pueden ser firmados por distintas partes.
  \item No es indispensable consignar el documento, lugar ni nombre.
  \item Escritura manuscrita o impresa.
  \item Redactados en cualquier idioma
\end{itemize}

Su valor probatorio depende del juez, que revisará la congruencia, precisión,
relaciones precedentes y confiabilidad del mismo.

\subsection{Acto administrativo}

Es una declaración unilateral realizada por la autoridad pública en ejercicio
de función administrativa que produce efectos jurídicos individuales de 
forma directa. Se caracteriza por ser:

\begin{description}
  \item[Legitimidad] un acto de este tipo se presume legítimo.
  \item[Ejecutividad] significa que el acto es obligatorio y exigible.
  \item[Ejecutoriedad] capacidad de la autoridad administrativa de obtener por sí
    el cumplimiento del acto.
  \item[Estabilidad] no puede ser dejado sin efecto por la propia administración
    cuando ya a sido creado o declarado.
  \item[Impugnabilidad] el acto puede ser recurrido, rechazado por quien esté 
    legitimado.
\end{description}

Todo acto administrativo además debe tener los siguientes requisitos esenciales:
competencia, causa, objeto, procedimientos, motivación y finalidad.

\subsection{Actas notariales}

El Art.310 del Código Civil y Comercial establece:

\space{0.3cm}
\begin{center}
\fbox{\begin{minipage}{0.85\textwidth}

\textit{Actas:} se denominan actas a los documentos notariales que tienen por 
objeto la comprobación de hechos.

\end{minipage}}
\end{center}
\vspace{0.3cm}

Luego, en el Art.311 del mismo se establecen las siguientes condiciones:

\begin{enumerate}
  \item Se debe constar lo que motiva la intervención del notario, y en su caso,
    la manifestación del requiriente respecto al interés propio de terceros
    que actúa. 
  \item No es necesaria la acreditación de personería ni la del interés del 
    tercero que alega el requiriente.
  \item No es necesario que el notario conozca o identifique a las personas con 
    las que trata.
  \item Las personas requeridas o notificadas deben ser previamente informadas 
    del cáracter en que interviene el notario.
  \item El notario puede practicar las diligencias sin la concurrencia del
    requiriente cuando por su objeto no sea necesario.
  \item No requieren unidad de acto ni de redacción.
  \item Pueden autorizarse aún cuando alguno de los interesados rehúse firmar.
\end{enumerate}

Estos documentos tienen \textbf{valor probatorio}, es decir:

\space{0.3cmu}
\begin{center}
\fbox{\begin{minipage}{0.85\textwidth}

El valor probatorio de las actas se circunscribe a los hechos que el notario 
tiene a la vista, a la verificación de su existencia y su estado. En cuando a
las personas, se circunscribe a su identificación si existe, y debe dejarse
constancia de las declaraciones y juicios que emiten. Las declaraciones deben 
referirse como mero hecho y no como contenido negocial".

\end{minipage}}
\end{center}
\vspace{0.3cm}



\end{document}
