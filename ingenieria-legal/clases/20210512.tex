\documentclass[../main.tex]{subfiles}

\begin{document}

\section{Construcción en inmueble ajenos}

\subsection{Muros linderos}

En principio se pueden ver dos tipos de muros:

\begin{description}
  \item[Encaballado] muro lindero que se asienta parcialmente en cada uno de los
    inmuebles colindantes.
  \item[Contiguo] son muros linderos que se asientan totalmente en uno de los
    inmuebles colindantes, de modo que el filo coincide con el límite.
\end{description}

También, según quien sea el dueño, se puede distinguir los \textbf{muros medianero},
que es común de los colindantes, o un \textbf{muro privativo}, que es un muro
que pertenece solo a uno de los colindantes.

\subsection{Construcción con materiales ajenos}

Si el dueño de un inmueble construye, siembra o planta con \textit{materiales ajenos}m
se puede distinguir en:

\begin{description}
  \item[De buena fe] adquiere los materiales, pero debe su valor.
  \item[De mala fe] adquiere los materiales, pero debe su valor y también los
    daños.
\end{description}

  Si la construcción es de un tercero, también distinguimos entre:

\begin{description}
  \item[De buena fe] los materiales pertenecen al dueño del inmueble, quien debe
    indemnizar el mayor valor adquirido.
  \item[De mala fe] el dueño del inmueble puede exigirle que reponga la cosa al
    estado anterior a su costa, a menos que la diferencia de valor sea importante,
    en cuyo caso debe el valor de los materiales y el trabajo, si no prefiere 
    abdicar su derecho con indenmización del valor del inmueble y del daño.
\end{description}

Por último, tenemos la situación donde se hace una construcción por un tercero,
con materiales o trabajo ajeno, quien efectúa el trabajo o quien provee los
materiales \textit{no tiene acción directa contra el dueño}, pero puede exigirle
al tercero.

\subsection{Invasión de inmueble colindante de buena fe}

Quien construye en su propio inmmueble, pero de buena fe, \textit{puede obligar}
\textit{a su dueño a resptear lo construido}, si éste no se opuso inmediatamente
de conocida la invasión.

Si es de buena fe, el inmueble colindante puede:

\begin{enumerate}
  \item Exigir la indemnización del valor de la parte invadida del inmueble.
  \item Reclamar su adquisición total si se menoscaba significativamente el 
    aprovechamiento normal del inmueble.
  \item En último caso, reclamar la disminución del valor de la parte no 
    invadida.
\end{enumerate}

Si el invasor de buena fe no quiere indemnizar, puede ser obligado a demoler lo
construido.

\subsection{Invsasión de mala fe}

Si el invsaor es de mala fe, y el dueño del fundo invadido se opuso inmediatamente
de conocida la invasión, éste puede pedir la demolición de lo construido. Sin
embargo, si resulta manifiestamente abusiva, el juez puede rechazar la petición
y ordenar la indemnización.


\end{document}
