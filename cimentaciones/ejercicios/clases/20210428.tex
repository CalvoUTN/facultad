\documentclass[../main.tex]{subfiles}

\begin{document}

\subsubsection{Continuación en siguiente clase}

\href{https://youtu.be/29KwPvQSee4}{Clase del 20210428}

Continuando con lo anterior, donde no voy a poder mostrar el modelo ya que no 
lo guardé como buen tonto, debemos terminar de modelar la platea de fundación.

Lo último que se hace es la elaboración de los apoyos elásticos cada $\SI{50}{cm}$,
y determinado la rigidez de cada uno de ellos dependiendo de la posición de donde
están, donde los interiores tienen, proporcionalmente 1, los de borde 1/2 y los
de esquina de 1/4.

Luego de esto podemos aplicar las cargas como sabemos. En este caso es interesante
que RAM Elements te permite generar la combinación, en nuestro caso, ASCE 7-05 LRFD,
que son las mismas consideradas por el reglamento CIRSOC. También existe uno 
análogo para los estados de servicio. En el caso mostrado se generan tres
combinaciones.



\end{document}

